% Energie-Sanierungsdokumentation (Deutsch) — Selbstenthaltend
\documentclass[11pt,oneside]{report}
\usepackage[utf8]{inputenc}
\usepackage[T1]{fontenc}
\usepackage{lmodern}
\usepackage{geometry}
\geometry{a4paper,margin=1in}
\usepackage[hidelinks]{hyperref}
\usepackage{bookmark}
\usepackage[ngerman]{babel}
\usepackage{textcomp}
\usepackage{newunicodechar}
\usepackage[hybrid]{markdown}
% Unicode-Zeichen abbilden
\newunicodechar{≥}{$\ge$}
\newunicodechar{≤}{$\le$}
\newunicodechar{°}{$^\circ$}
\newunicodechar{·}{\ensuremath{\cdot}}
\newunicodechar{−}{-}
\newunicodechar{–}{--}
\newunicodechar{—}{---}
\newunicodechar{↔}{$\leftrightarrow$}
\newunicodechar{→}{$\to$}
\newunicodechar{≈}{$\approx$}
\newunicodechar{²}{\ensuremath{^2}}
\newunicodechar{Δ}{$\Delta$}
\newunicodechar{₂}{$_2$}
\newunicodechar{⁻}{$^{-}$}
\newunicodechar{€}{\texteuro{}}

\title{Energie-Sanierungsdokumentation\\Doppelpuffer, WRG, Regelung, Förderung}
\author{Projekt: Reihenmittelhaus Baujahr 1972}
\date{\today}

\begin{document}
\maketitle
\setcounter{secnumdepth}{3}
\setcounter{tocdepth}{3}
\tableofcontents
\clearpage

\chapter{Abkürzungen}
\begin{markdown}
- L/W‑WP — Luft‑Wasser‑Wärmepumpe
- WP — Wärmepumpe
- TWW — Trinkwarmwasser
- FriWa — Frischwasserstation (Durchlauf‑TWW über Plattenwärmetauscher)
- UFH — Fußbodenheizung
- WRG — Wärmerückgewinnung
- PHE/PWT — Plattenwärmetauscher
- PV — Photovoltaik
- HA — Home Assistant (Open‑Source‑Automation)
- COP — Leistungszahl (Coefficient of Performance)
- SoC — Ladezustand (Batterie)
- SPD — Überspannungsschutzgerät (Surge Protective Device)
- MAG — Membran‑Ausdehnungsgefäß
- RLA — Rücklaufanhebung (z. B. ≥ 60 °C)
- TAS — Thermische Ablaufsicherung
- VDI — Verein Deutscher Ingenieure
- DIN — Deutsches Institut für Normung
- EN — Europäische Norm
- KfW — Kreditanstalt für Wiederaufbau
- BAFA — Bundesamt für Wirtschaft und Ausfuhrkontrolle
- BEG — Bundesförderung für effiziente Gebäude
- iSFP — individueller Sanierungsfahrplan
- BzA — Bestätigung zum Antrag
- BnD — Bestätigung nach Durchführung
- WMZ — Wärmemengenzähler
- RCD — Fehlerstromschutzschalter
- LS — Leitungsschutzschalter (MCB)
- ATS — Automatischer Netzumschalter (Automatic Transfer Switch)
\end{markdown}

\chapter{Projektüberblick}
\begin{markdown}
# Projektüberblick

- Gebäude: Reihenmittelhaus (~113 m²), Baujahr ~1972, EG + DG + Keller.
- Belegung: 3 Personen.
- Energie (2023):
  - Raumwärme: ~9.824 kWh/a (~87 kWh/m²·a), Fernwärme.
  - TWW‑Volumen: ~16,86 m³ (≈881 kWh/a thermisch, ΔT≈45 K).
  - Strom: ~2.671 kWh/a.
- Ziele:
  - Effizienzhaus 85 + Erneuerbare‑Energien‑Klasse (≥65 % erneuerbare Wärme/Kälte) erreichen.
  - Hohen PV‑Eigenverbrauch, Resilienz durch Batteriespeicher und Unterverteilung für wichtige Stromkreise.
  - Einfache, effiziente Bedienung, sichere Regelung, saubere Inbetriebnahme.
  - Sanftes hydronisches Kühlen ohne Klimageräte (UFH + Taupunkt‑Schutz).

## Schlüsselentscheidungen

- Doppelpuffer‑System:
  - TWW‑Heizspeicher (200–300 L) ganzjährig heiß (55–60 °C), speist FriWa (kein Trinkwasserspeicher).
  - Heiz-/Kühlspeicher (800–1.000 L) im Winter warm, im Sommer kalt (Ziel 16–18 °C für Nachtladung).
- Wärmeerzeuger: Reversible Luft‑Wasser‑Wärmepumpe (L/W‑WP, R290), monovalent nach Heizlast; wasserführender Holzofen als Komfort/Backup.
- Verteilung: Wassergeführte Fußbodenheizung (UFH) in Hauptzonen; keine Gebläsekonvektoren, keine AC.
- Lüftung: Dezentrale Einzelraumgeräte (6–8 Stück) gemäß DIN 1946‑6.
- Steuerung: Harte Sicherheits‑/Regeltechnik + Taupunktlogik; Home Assistant (HA) zur PV‑Orchestrierung/Monitoring.

## Zielwerte

- Hülle (typisch EH85):
  - Kellerdecke U ≤ 0,25 W/(m²·K)
  - Dach/Oberste Decke U ≤ 0,14 W/(m²·K)
  - Fenster Uw ≤ 0,90 W/(m²·K)
  - Luftdichtheit n50 ≤ 1,5 h⁻¹ (Blower‑Door vorher/nachher)
- Hydronik:
  - Heizen: Auslegung VL ≤ 35 °C; lange Verdichterlaufzeiten.
  - Kühlen: VL ≥ Taupunkt + 2 K, typ. 19–21 °C.
  - UFH‑Kühlleistung: ~10–25 W/m² (≈1,1–2,8 kW gesamt).

## Erwartete Wirkungen (grobe Ordnung)

- Hülle + Wärmebrücken: ~25–40 % weniger Heizenergie ggü. Ist‑Zustand.
- PV ~5,8–8 kWp: ~5,5–8,5 MWh/a; mit ~10 kWh Speicher 50–70 % Eigenverbrauch realistisch.
- Hydronisches Kühlen: angenehme Grundkühlung; Feuchte über Lüftung/Taupunktlogik geführt (bei Extremwetter ggf. mobiler Entfeuchter).

## Randbedingungen

- Offenes Kellertreppenhaus: Luftdichte, transparente Abtrennung + Kellerdeckendämmung.
- Loggia/Eingangsbereich: ausgeprägte Wärmebrücken; Optionen siehe 05_envelope-airtightness.md.
- Keine zentrale Lüftung möglich: Einzelraumgeräte mit akustischer und strömungstechnischer Planung.
\end{markdown}

\chapter{Verbrauch und Kosten (2021--2023)}
\begin{markdown}
Konzentriert auf Mengen und Kosten aus den Fernwärme‑Rechnungen (Quellen siehe unten).

## Kennzahlen

- 2021
  - Raumwärme: 14.542 kWh — 2.158,76 EUR
  - TWW: 37,86 m³ — 341,11 EUR
  - Summe (Wärme + TWW): 2.499,88 EUR
- 2022
  - Raumwärme: 10.898 kWh — 1.917,08 EUR
  - TWW: 23,00 m³ — 232,04 EUR
  - Summe (Wärme + TWW): 2.149,12 EUR
- 2023
  - Raumwärme: 9.824 kWh — 3.197,43 EUR
  - TWW: 16,86 m³ — 445,41 EUR
  - Summe (Wärme + TWW): 3.642,84 EUR
  - CO2‑Info (aus Rechnung): Faktor 0,251 kg CO2/kWh; Emissionen 2.961 kg; CO2‑Kosten 95,04 EUR

## Quellen

- 2021: `infosAndStuff/2021_10000027695_818349.pdf`
- 2022: `infosAndStuff/2022_10000027695_818349.pdf`
- 2023: `infosAndStuff/2023_10000027695_818349.pdf`
\end{markdown}

\chapter{Entwurfsarchitektur — Doppelpuffer}
\begin{markdown}
# Entwurfsarchitektur — Doppelpuffer

Hydraulisches Konzept mit zwei Speichern: ein separater, ganzjährig heißer TWW‑Heizspeicher (speist eine Frischwasserstation, FriWa) sowie ein Heiz‑/Kühlspeicher, der im Winter warm und im Sommer kalt betrieben wird. Keine Klimageräte oder Fan‑Coils; Kühlung über UFH mit Taupunkt‑Schutz.

## Einlinienschema (ASCII)

\end{markdown}
\begin{Verbatim}[fontsize=\small]
           [PV + Batterie + Unterverteilung Wichtige Stromkreise]
                              |
                        [Netzversorgung]
                              |
                         [L/W-WP R290]
                              |
                     (optional Platten-WT)
                              |
                     +--------+---------+
                     |                  |
          [Heiz-/Kuehlspeicher]       [TWW-Heizspeicher]
            800-1000 L, schichtend      200-300 L, 55-60 C
                     |                  |
                     |                  +--> [FriWa] --> TWW zu Zapfstellen
                     |                           ^
                     |                           |
               +-----+----+                 Kaltwasser
               |          |
            3-Wege      Ruecklauf
            Mischer       UFH
               |          ^
               v          |
            [UFH VL]  --> Loops --> [UFH RL]

 [Wasserfuehrender Ofen] --(Ruecklaufanhebung, Sicherheit)--> Puffer oben
\end{Verbatim}
\begin{markdown}

Hinweise:
- Der TWW‑Heizspeicher enthält Heizungswasser (kein Trinkwasser). Die FriWa erwärmt Trinkwasser im Durchlauf über einen Plattenwärmetauscher.
- Der Heiz-/Kühlspeicher dient als hydraulische Weiche und Energiespeicher. Im Sommer nachts kühlen, tagsüber nutzen.
- Optionaler Plattenwärmetauscher (PHE) zwischen WP und Hauskreis ermöglicht Glykol auf WP‑Seite und schützt die Wasserqualität.

## Betriebsarten

- Winter
  - Wettergeführter Betrieb; WP belädt den Heiz-/Kühlspeicher (Mitte) für den Heizbetrieb; Auslegung VL ≤ 35 °C.
  - Mittags (PV‑Fenster) hat TWW Priorität: TWW‑Speicher auf 55–60 °C laden, danach Rückkehr zum Heizen.
  - Ofen lädt Puffer oben (Rücklaufanhebung ≥ 60 °C); thermische Ablaufsicherung und Notkühlung gemäß EN 303‑5.
  - HA orchestriert Prioritäten/Meldungen; Sicherheits- und Grenzwerte sind hart verdrahtet.

- Sommer
  - Nacht: WP kühlt den Heiz-/Kühlspeicher auf ~16–18 °C.
  - Tag: UFH‑Vorlauf wird auf ≥ (Taupunkt + 2 K) begrenzt, typ. 19–21 °C (Kondensationsschutz).
  - TWW‑Speicher bleibt ganzjährig heiß; mittags per PV laden.

- Übergangszeit
  - Minimale Puffersollwerte; TWW mittags; Nachtkühlung bei Hitzeperioden.

## Sollwerte, Grenzen, Sensorik

- TWW‑Heizspeicher: 55–60 °C; Verbrühschutz an Entnahmestellen.
- Heiz-/Kühlspeicher Winter: so, dass VL ≤ 35 °C erreicht wird.
- Heiz-/Kühlspeicher Sommer: Nachtziel 16–18 °C.
- UFH im Kühlbetrieb: ≥ (Taupunkt + 2 K). Beispiel: 26 °C/60 % r.F. → Taupunkt ≈ 17,8 °C → VL ≥ 20 °C.
- Sensorik (Minimum):
  - Speicher oben/Mitte/unten (beide Speicher)
  - UFH: VL/RL; optional Oberflächensensor am Verteiler
  - Raumklima: mind. ein T/r.F.-Sensor pro Etage (Taupunkt)
  - Wärmemengen: WP→Puffer, ggf. FriWa‑Primär

## Dimensionierung (Richtwerte)

- WP: nach EN 12831; typisch 4–6 kW @ −10 °C; Gerät 6–10 kW Klasse mit guter Modulation/Leisebetrieb.
- Heiz-/Kühlspeicher: 800–1000 L, 100–150 mm Dämmung, geringe Δp‑Anschlüsse, 3 Tauchhülsen.
- TWW‑Heizspeicher: 200–300 L, 3 Tauchhülsen, 55–60 °C.
- FriWa: 25–35 kW Platten‑WT‑Modul mit variabler Primärpumpe, Durchflusssensor, elektronischer Auslauftemperaturregelung; Trinkwasserseite mit Filter/Service.
- Ventile/Pumpen: Hocheffizienzpumpen; 3‑Wege‑Mischer (UFH) mit fester Taupunkt‑Begrenzung; motorische Absperrungen für Sommer/Winter; Rückflussverhinderer, Strangregulierungen.
- Optionaler PHE: auf volle Leistung mit kleinem Annäherungs‑ΔT; Glykol WP‑seitig; Luft/Schlammabscheider.
- Ausdehnungsgefäße: auf Gesamtvolumen ausgelegt (Speicher + Rohrnetz + Heizflächen).

## Sicherheit und Wasserqualität

- Ofenkreis: Rücklaufanhebung (RLA, ≥ 60 °C), thermische Ablaufsicherung (TAS), schwerkrafttaugliche Notkühlung, ausreichendes Membran‑Ausdehnungsgefäß (MAG); EN 303‑5.
- System: Sicherheitsventile (i. d. R. 3 bar), Luft/Schlammabscheider, VDI 2035‑gerechtes Wasser; warme Leitungen dämmen, kalte diffusionsdicht inkl. Kondensatableitung.
- Elektro: Unterverteilung für wichtige Stromkreise (HP‑Freigabe, Pumpen, Regler, Netzwerk, Kühlung, ausgewählte Stromkreise).

## Regelungszuordnung

- Primär/Sicherheit: WP‑Regler; Hydraulikregler für Mischer, Pumpen, Temperaturgrenzen, TWW‑Priorität; harte Verriegelungen (nicht HA‑abhängig).
- Orchestrierung (HA): PV‑Zeitfenster (z. B. TWW mittags), Saisonumschaltung, Meldungen (Ofen, Filterservice, r.F./T‑Abweichungen), Logging (Puffer, r.F., Energie, COP‑Proxy).
\end{markdown}

\chapter{Trinkwarmwasser (TWW)}
\begin{markdown}
# Trinkwarmwasser (TWW) — FriWa + TWW‑Heizspeicher

Separater TWW‑Heizspeicher (Heizungswasser) mit ~55–60 °C versorgt eine Frischwasserstation (FriWa). Es gibt keinen Trinkwasserspeicher; TWW wird im Durchlauf erwärmt.

## Begründung

- Hygiene: Kein stehendes Trinkwarmwasser → minimiertes Legionellenrisiko; Verbrühschutz an Zapfstellen.
- Sommerkompatibilität: TWW‑Speicher bleibt heiß, Heiz-/Kühlspeicher kann kalt bleiben (UFH‑Kühlung).
- PV‑Synergie: TWW‑Ladung mittags nutzt PV‑Überschuss, reduziert Abendlast.

## Hydraulik

- TWW‑Heizspeicher (200–300 L): 55–60 °C ganzjährig; oben/Mitte/unten Sensoren; gute Dämmung.
- FriWa (25–35 kW): Platten‑WT, variable Primärpumpe, Durchflusssensor, elektronische Auslauftemperatur; Trinkwasserseite mit Filter/Absperrungen.
- WP‑Priorität: Zeitbasiert (mittags) und temperaturgeführt (wenn Speichertemperatur < Soll) zuerst TWW, sonst Heiz-/Kühlspeicher.

## Regelung/Sollwerte

- TWW‑Soll: 55–60 °C (Wasserhärte/Komfort beachten). Verbrühschutzmischer einsetzen.
- PV‑Fenster: Bei hohem Ladezustand (SoC) und PV‑Überschuss TWW laden; sonst zeitversetzt/off‑peak.
- Stillstandsverluste: Speicher gut dämmen; ggf. Nachtabschaltung, wenn Bedarfsmuster es zulässt.

## Dimensionierung/Leistung

- Speicher: 200–300 L für 3 Personen (Duschen, gelegentlich Baden); eher 300 L bei parallelen Zapfungen.
- FriWa: 25–35 kW liefern typ. 10–16 L/min bei 40–45 °C (abhängig von Primärtemperatur/Kennlinien).
- Annäherung ΔT: Primärtemperatur mit Reserve zur Auslauftemperatur; FriWa‑Regelung fein abstimmen.

## Wasserqualität/Wartung

- Trinkwasserseite: Filter/Sieb; Inspektion/Entkalkung gemäß Wasserhärte.
- Primärseite: VDI 2035 (Leitfähigkeit/Härte) zum Schutz von WT, Pumpen, Ventilen.
- Service: Absperrungen, Entleerungen, Messstellen vorsehen.

## Inbetriebnahme‑Checks

- Stabiler TWW‑Auslauf 2–16 L/min ohne Überschwingen.
- Verbrühschutz prüfen.
- TWW‑Ladezeiten/Temperaturverläufe loggen und PV‑Timing verifizieren.
\end{markdown}

\chapter{Lüftung — Dezentrale Einzelraum-WRG}
\begin{markdown}
# Lüftung — Dezentrale Einzelraum‑WRG (DIN 1946‑6)

Da keine zentrale Lüftung möglich ist, kommen 6–8 dezentrale Geräte zum Einsatz. Sie stellen den Grundluftwechsel mit Wärmerückgewinnung (WRG) sicher, besitzen Boost‑Funktionen und Sommerbypass.

## Prinzipien

- Bilanz/Abdeckung: Kontinuierlicher Grundluftwechsel in Wohn‑/Schlafräumen; stärkere Abfuhr/Boost in Feuchträumen.
- Akustik: Leise im Grundbetrieb, Nachtmodus; akustische Einbauteile.
- Einfacher Einbau: Kernbohrung 160–200 mm, leichtes Gefälle nach außen (Kondensat), Filterzugang.

## Gerätetypen

- Doppellüfter‑Dauerbetrieb (bevorzugt): gleichzeitige Zu‑/Abluft über kleinen Gegenstromkern; stabile Luftbilanz.
- Wechselbetrieb (Paarweise): Keramikspeicher; paarweise Montage zur Bilanzierung.

## Platzierung/Luftwege

- Feuchträume (Bad/WC/Küche): Geräte mit Boost 40–60 m³/h; ggf. Fettfilter.
- Wohn‑/Schlafräume: Grundvolumen 15–30 m³/h je Raum; Schlafräume mit Nachtmodus.
- Türunterkanten/Überströmöffnungen; Kurzschlussströmung vermeiden.
- Außenhauben mit Wetterschutz/Insektenschutz; Fassadenakustik beachten.

## Steuerung

- Lokal: Taster für Boost (Bad), automatische Boosts (r.F./CO₂), Nachtmodi.
- Zentrale Übersicht (optional): potentialfreie Kontakte oder Modbus/IP‑Gateway zur HA‑Integration (Monitoring, nicht sicherheitskritisch).
- Sommerbypass: aktivieren, um unerwünschte Wärmerückgewinnung in der Kühlperiode zu vermeiden; Querlüftung nutzen.

## Einbau/Inbetriebnahme

- Kernbohrungen mit leichtem Gefälle nach außen; luftdichte Hüll‑Anbindung.
- Volumenströme gemäß DIN 1946‑6 einstellen, messen, protokollieren.
- Filter: Wartungsintervalle definieren; Ersatzfiltersätze vorhalten; Service‑Hinweise in HA.

## Zusammenspiel mit hydronischem Kühlen

- WRG hilft bei Feuchtestabilisierung, entfeuchtet aber nur begrenzt; r.F. pro Etage überwachen.
- Bei anhaltend >60 % r.F. in Hitzephasen temporär Entfeuchter einsetzen (weiterhin „keine AC“).
\end{markdown}

\chapter{Gebäudehülle, Luftdichtheit, Wärmebrücken}
\begin{markdown}
# Gebäudehülle, Luftdichtheit, Wärmebrücken

Maßnahmen für Dämmung, Luftdichtheit und die neuralgischen Wärmebrücken (Loggia, offener Eingangsbereich, offenes Kellertreppenhaus).

## Zielwerte (EH85‑nah)

- Kellerdecke U ≤ 0,25 W/(m²·K)
- Dach/Oberste Decke U ≤ 0,14 W/(m²·K)
- Fenster Uw ≤ 0,90 W/(m²·K) mit warmer Kante
- Luftdichtheit n50 ≤ 1,5 h⁻¹ (Blower‑Door vorher/nachher)

## Maßnahmen

- Kellerdecke (muss): Platten oder Einblasdämmung; Durchdringungen abdichten; Anschlussdetails durchgängig.
- Dach/Oberste Decke (muss): Aufdachdämmung oder oberste Decke; luftdichte Ebene konsequent; Details an Durchdringungen.
- Fenster (muss): Dreifachverglasung, luftdichte Montage mit Bändern/Kompribändern; Laibungsdämmung prüfen.
- Fassade (optional, empfohlen): WDVS außen, falls möglich; sonst kapillaraktive Innendämmung (z. B. CaSi/Holzfaser) an kritischen Innenflächen; Feuchteführung planen (keine Dampfsperrfallen).
- Luftdichtheitspaket (muss): Professionelle Abdichtung, Blower‑Door vor/nach Sanierung.

## Offenes Kellertreppenhaus

- Luftdichte, transparente Abtrennung (Glas/Schiebetür) auf Keller‑ oder EG‑Ebene; Lichtbezug erhalten; Dichtprofile.
- In Kombination mit Kellerdeckendämmung verringert Kamineffekt.

## Loggia (innenliegender Balkon)

- Wintergarten: Dreifachglas, thermisch getrennte Rahmen, gedämmte Brüstungen/Laibungen, außenliegende Verschattung; beste Behaglichkeit; Genehmigung/Statik beachten.
- Zielgerichtete Wärmebrücken‑Sanierung: kapillaraktive Innendämmung an Knoten; ggf. außen ergänzen; geringer, aber günstiger Effekt.

## Offener Eingang/überdeckter Vorbereich

- Verglaster Windfang: gedämmte Rahmen und Außentür; großer Nutzen gegen Infiltration/Kälte; ggf. Genehmigung.
- Unterseitige Dämmung: starre PIR/EPS‑Platten; Seitenanschlüsse luftdicht/gedämmt.

## Verifikation

- Thermografie nach Fertigstellung zur Restwärmebrückensuche.
- Blower‑Door (n50 ≤ 1,5 h⁻¹), Leckagenachweise.
\end{markdown}

\chapter{Regelung, Sensorik, Monitoring}
\begin{markdown}
# Regelung, Sensorik, Monitoring

Sicherheits‑ und Primärregelung sind hart verdrahtet; Home Assistant (HA) dient nur zur Orchestrierung/Visualisierung. Taupunkt‑Schutz ist Pflicht.

## Sicherheit/Primärregelung (hart)

- WP‑Regler: Verdichterschutz, Abtauung, Vor-/Rücklauftemperaturlimits.
- Hydraulikregler: 3‑Wege‑Mischer UFH, Pumpen, Puffertemperaturgrenzen, TWW‑Priorität.
- Taupunktabschaltung: harte Begrenzung UFH‑VL ≥ (Taupunkt + 2 K), unabhängig von HA.
- Ofen: Rücklaufanhebung (≥ 60 °C), thermische Ablaufsicherung, schwerkraftgeeignete Notkühlung, MAG/PRV.

## Sensorik/Messtechnik

- Speicher: oben/Mitte/unten an TWW- und Heiz-/Kühlspeicher.
- UFH: VL/RL; optional Oberflächensensor am Verteiler.
- Raum: mind. ein T/r.F. je Etage (Taupunktberechnung).
- Energiemessung: Wärmemengen WP→Puffer, optional FriWa‑Primär; Strom‑Unterzähler WP.
- Durchfluss/Druck: nach Bedarf (FriWa/Strangregulierung), Entlüftungen/Spülstellen.

## Orchestrierung (HA o. ä.)

- PV‑Zeitfenster: TWW‑Ladung mittags; im Winter optional Puffer‑Vorwärmung bei PV‑Überschuss.
- Saisons: Winter (Heizen), Sommer (Kühlen + TWW), Übergang (minimal).
- Meldungen: Ofen‑Hinweise, Filterservice (WRG/FriWa), r.F./T‑Alarme, Fehlerrelais WP.
- Logging: Puffer‑T, UFH VL/RL, r.F./T innen, WP‑Leistung, WMZ; Leistungszahl (COP)‑Proxy möglich.

## Beispiel Taupunktlogik

- Eingang: T innen, r.F. innen → Taupunkt; gemessener UFH‑VL.
- Limit: UFH‑VL‑Soll = max(Heizkurvenbedarf, Taupunkt + 2 K), absolut min. typ. 19–21 °C.
- Sperre: Bei VL < (Taupunkt + 2 K) oder Feuchte am Verteiler: Mischer schließen/Pumpe stoppen bis sicher.

## Elektro/Resilienz

- Unterverteilung wichtige Stromkreise: WP‑Freigabe, Pumpen, Hydraulikregler, HA‑Host, Netzwerk, Kühlgerät, ausgewählte Licht/Steckdosen.
- Überspannungsschutz (SPD) für PV, Speicher, WP, MSR.
- Nachtmodus: reduzierte Schallemission WP/WRG.
\end{markdown}

\chapter{Förderung, Konformität, Dokumentation}
\begin{markdown}
# Förderung, Konformität, Dokumentation

Strategie zur Förderung (Stand: konzeptionell). Verbindliche Prüfung stets mit Energie‑Effizienz‑Experte (EEE) und Bank vor Beauftragung.

## Primär — KfW 261 (Wohngebäude – Kredit)

- Ziel: Effizienzhaus 85 + Erneuerbare‑Energien‑Klasse (≥65 % erneuerbare Wärme/Kälte).
- Inhalt: Wärmepumpe, Speicher (Heiz/Kühl + TWW), FriWa, Hydraulik‑Peripherie, Sicherheit, System‑Elektro; ggf. wasserführender Ofen.
- Tilgungszuschuss: typ. 10 % auf finanzierten Anteil nach BnD durch EEE.
- Wichtig: Keine Doppelförderung derselben Kostenposition außerhalb des EH‑Topfs.

## Ergänzend — Einzelmaßnahmen (BAFA/BEG)

- Hülle (Keller/Dach/Fenster/Fassade), dezentrale Lüftung, Mess‑/Steuer‑/Regelung (MSR), UFH‑Verteilung.
- Sätze: 15 % bzw. 20 % mit iSFP.
- Regel: Keine Doppelansetzung, wenn bereits im EH‑Kredit enthalten.

## Weitere Finanzierung

- Ergänzungskredite (z. B. KfW 358/359) für Einzelmaßnahmen.
- PV/Batterie: 0 % MwSt.; separat finanzierbar (z. B. KfW 270) oder Hausbank.

## Prozess/Timing

- EEE frühzeitig einbinden; „Bestätigung vor Antrag“ (BzA) vor Auftragsvergabe bzw. mit aufschiebender Bedingung.
- Saubere Kostentrennung: EH‑Topf vs. Einzelmaßnahmen definieren.
- „Bestätigung nach Durchführung“ (BnD) zur Zuschussgutschrift/Abschluss.

## Nachweise/Doku

- Berechnungen: EN 12831 (Heizlast), Lüftungskonzept nach DIN 1946‑6, Taupunktstrategie, Wärmebrückendetails.
- Inbetriebnahmeprotokolle: Blower‑Door (vor/nach), hydraulischer Abgleich, Wasserqualität (VDI 2035), Druckprobe, Wärmemengenzähler, WP‑Inbetriebnahme.
- Sicherheit: Ofen EN 303‑5 (RLA, TAS, Notkühlung), Elektro‑SPDs, Unterverteilung für Notbetrieb.
- Monitoring: erste Betriebsperiode (T, r.F., WMZ/EMZ) zur Performancevalidierung.

## Hinweise

- Programme ändern sich; Konditionen/Caps/Fristen aktuell prüfen.
- Eine „Single Source of Truth“ für Kostenzuordnung/Belege vereinfacht Prüfungen.
\end{markdown}

\chapter{Inbetriebnahme und Abnahme}
\begin{markdown}
# Inbetriebnahme und Abnahme

Checkliste für sichere Inbetriebnahme und förderkonforme Nachweise.

## Vor‑Inbetriebnahme

- Mechanik:
  - Spülung/Druckprobe bestanden; dicht.
  - MAG dimensioniert und voreingestellt (Gesamtvolumen Speicher+Anlage).
  - Luft/Schlammabscheider gesetzt; Entlüftungen zugänglich.
  - Dämmung vollständig: warm gedämmt; kalt diffusionsdicht inkl. Kondensatführung.
- Elektro:
  - Unterverteilung wichtige Stromkreise verdrahtet; Beschriftung; RCD/LS geprüft.
  - SPDs installiert und mit PV/Batterie koordiniert.
  - Sensorik/Aktoren beschriftet; Not‑Stop dokumentiert.
- Wasserqualität:
- VDI‑Richtlinie 2035 (Verein Deutscher Ingenieure) konformes Wasser; Leitfähigkeit/Härte protokolliert.
- Sicherheit (Ofen):
  - Rücklaufanhebung, thermische Ablaufsicherung, Notkühlweg, Schornsteinfreigaben geprüft.

## Regelung/Sensorik prüfen

- Speicher: oben/Mitte/unten plausibel; Flussrichtungen geprüft.
- UFH: VL/RL‑Sensoren, 3‑Wege‑Mischer‑Richtung, Pumpendrehzahl.
- Taupunktlogik: Feuchte‑Szenario testen; Begrenzung/Abschaltung verifizieren.
- WP‑Regler: Heizkurve/Min‑Max VL/Antitakt eingestellt.
- FriWa: Auslauftemperatur stabil, 2–16 L/min; Verbrühschutz verifiziert.

## Funktionstests

- Heizen (Winter‑Simulation):
  - Lange Verdichterläufe; Schichtung sichtbar; Ziel‑VL erreicht.
  - TWW‑Mittagspriorität auf 55–60 °C; danach Heizen.
- Kühlen (Sommer‑Simulation):
  - Nachtkühlung Puffer ~16–18 °C; Tag VL = Taupunkt + 2 K; keine Kondensation.
- Ofen‑Einbindung:
  - Puffer oben aufladen; WP‑Leistungsreduktion bei hoher Top‑Temperatur.

## Hydraulischer Abgleich

- UFH‑Kreise messen, einstellen, protokollieren.
- WMZ prüfen (Einbaurichtung, Impulse fürs Logging).

## Lüftung (Dezentrale WRG)

- Raumweise Volumenströme einstellen; Boost; Sommerbypass.
- Filter eingesetzt; Wartungsplan; Akustik prüfen.

## Abnahmedokumente

- Blower‑Door (vor/nach), n50 ≤ 1,5 h⁻¹ (Ziel).
- Protokolle hydraulischer Abgleich, WP/FriWa‑Inbetriebnahme.
- VDI 2035, Druckprüfungen, Elektro (RCD/Isolationsmessung/SPD).
- Schemata: finales Hydraulik‑Schema, I/O‑Plan, Sensorliste.
- Sicherheit: EN 303‑5 Nachweise, Schornsteinfeger‑Abnahmen, Notfall‑Prozeduren.
- Monitoringplan: zu erfassende Daten, Aufbewahrungsdauer (mind. 1. Saison).
\end{markdown}

\chapter{Projektphasen und Reihenfolge}
\begin{markdown}
# Projektphasen und Reihenfolge

Pragmatische Abfolge zur Minimierung von Rückbau, passgenauer Förderung und qualitativem Hochlauf.

## Phasen

1) Voruntersuchungen
- Gefahrstoffe (Bj. 1972): Asbest/PCB/alte Mineralwolle.
- Statik Flachdach für PV; Aufstellort/Schall; Genehmigungen (Loggia/Windfang).

2) Konzept/Berechnungen
- EN 12831 Heiz‑ (und ggf. Kühl‑)last; DIN 1946‑6 Lüftungskonzept.
- Wärmebrückendetails Loggia/Eingang; Taupunktstrategie UFH.
- Kostenzuordnung EH‑Topf vs. Einzelmaßnahmen.

3) Förderanträge
- BzA durch EEE; Bankgespräch.
- KfW 261 (EH); ergänzende BAFA/BEG‑Einzelmaßnahmen und evtl. Zusatzkredite.

4) Hülle/Luftdichtheit
- Kellerdecke, Dach, Fenster; optional Fassade.
- Treppenhaus‑Abtrennung; Loggia/Windfang (falls genehmigt).
- Blower‑Door (Zwischenmessung) als QS.

5) Technik
- UFH‑Verteilung; Abgleich‑Bereitschaft.
- WP, Speicher (TWW + Heiz/Kühl), FriWa, Pumpen/Ventile, Sensorik, Unterverteilung.
- Dezentrale WRG inkl. Einregulierung.

6) PV/Batterie
- PV/Batterie; Unterverteilung integrieren; SPD.

7) Inbetriebnahme/Feinabstimmung
- Spülung/Druckprobe/VDI 2035; Parametrierung; Taupunkt‑Test; Ofen‑Sicherheitstests.
- Abgleich‑Protokoll; Datenlogging.

8) Abschluss
- BnD (EEE); Zuschüsse/Tilgungszuschuss.
- Erste Saison: Monitoring/Optimierung; ggf. Thermografie.
\end{markdown}

\chapter{Stückliste}
\begin{markdown}
# Stückliste (Spezifikationsklassen)

Markenneutrale Auswahl mit Größenordnungen. Endgültige Auswahl nach Detailberechnung und Installationsstandard.

## Erzeugung/Speicher

- Reversible L/W‑WP (R290), 6–10 kW Klasse, Monoblock/Split; Leise‑Kit; Nachtmodus.
- Optionaler Plattenwärmetauscher (WP↔Haus), volle Leistung bei kleinem ΔT; Glykol WP‑seitig.
- Heiz-/Kühlspeicher: 800–1000 L, schichtend, 100–150 mm Dämmung, 3× Tauchhülsen.
- TWW‑Heizspeicher: 200–300 L, hohe Dämmung, 3× Tauchhülsen.
- FriWa‑Modul: 25–35 kW Platten‑WT, variable Primärpumpe, Durchflusssensor, Auslauftemp‑Regelung, Servicearmaturen, Trinkwasserfilter.
- Wasserführender Ofen (optional): Leistung passend; Rücklaufanhebung (≥60 °C), TAS, Notkühlweg; Schornstein‑Zubehör.

## Hydraulische Peripherie

- Pumpen: ECM für WP‑, Puffer‑, UFH‑ und FriWa‑Kreise.
- Ventile: 3‑Wege‑Mischer (UFH), motorische Absperrungen (Saison), RVs, Strangregulierventile, Füll/Entleer.
- Abscheider: Luft/Schlamm; optional Magnetit.
- MAG (Membran‑Ausdehnungsgefäß): auf Gesamtvolumen ausgelegt, mit Serviceventil/Manometer.
- Sicherheit: Sicherheitsventile (PRV, pressure relief valves, typ. 3 bar), Manometer, Automatikentlüfter; Kondensatfallen an Kaltleitungen.

## Verteilung

- UFH‑Verteiler/Schleifen: sauerstoffdiffusionsdicht (PEX/MLCP), Verteilerkästen, Durchflussmesser, Stellantriebe (bei Zonenbetrieb).
- Rohrdämmung: warm gemäß Norm; kalt diffusionsdicht; Verteiler/Armaturen dämmen.

## Lüftung (dezentral)

- 6–8 Einzelraum‑WRG‑Geräte (Doppellüfter oder Paar‑Wechsel), Wandeinbauhülsen, Außenhauben, Akustikelemente, Filter.
- Steuerung: Boost‑Taster, r.F./CO₂‑Sensoren (sofern unterstützt), optional Gateway zur HA‑Integration.

## Regelung/Sensorik/Elektro

- Hydraulikregler: Mischer/Pumpen mit Taupunkt‑Eingang und Abschaltung.
- Sensoren: Speicher T oben/Mitte/unten, UFH VL/RL, Raum T/r.F. je Etage, optional Verteiler‑Oberfläche.
- Energiemessung: WMZ WP und optional FriWa‑Primär; Strom‑Unterzähler WP.
- Elektro: Unterverteilung wichtige Stromkreise, Umschalter/Ersatzstrom, SPD, Beschriftung, Installationsmaterial.
- HA‑Host: zuverlässige Kleinrechner‑Plattform; Netzwerk; optional USV.

## Wasseraufbereitung/Service

- VDI 2035‑Einheit/Chemie; Testkit (Leitfähigkeit/Härte); Befüll/Entleer.
- Filter/Siebe: Trinkwasserfilter FriWa; Strainer primär; Ersatzfilter.
- Service: Absperrungen, Entleerungen, Tauchhülsen/Messstellen, Zugänge.
\end{markdown}

\chapter{Risiken und Gegenmaßnahmen}
\begin{markdown}
# Risiken und Gegenmaßnahmen

Wesentliche technische/projektbezogene Risiken und praxistaugliche Lösungen.

## Hydronisches Kühlen/Kondensation

- Risiko: UFH‑Oberflächen/Verteiler unter Taupunkt → Tauwasser/Schäden.
- Maßnahmen: harte Taupunkt‑Limitierung (Mischer), diffusionsdichte Kältedämmung, Kondensatführung, r.F.‑Überwachung je Etage, Kühlung bei >60 % r.F. temporär sperren.

## Systemkomplexität

- Risiko: Doppelpuffer + FriWa + Ofen erhöhen Komponenten/Regelung.
- Maßnahmen: Sicherheitsverriegelungen in dedizierten Reglern (nicht HA), klare Betriebsmodi, strukturierte Inbetriebnahme, Beschriftung, Servicedoku.

## Schall (Reihenhaus)

- Risiko: Außengerät/WRG stören Bewohner/Nachbarn.
- Maßnahmen: akustisch günstiger Standort/Schirm, entkoppelte Montage, Nachtmodus, WRG‑Akustikmaßnahmen.

## Wärmebrücken (Loggia, Eingang)

- Risiko: Energieverlust, kalte Oberflächen, Feuchte.
- Maßnahmen: bevorzugt Wintergarten/Windfang; alternativ kapillaraktive Innendämmung + außenliegende Verschattung; Thermografie nach Fertigstellung.

## Wasserqualität/Verkalkung

- Risiko: Platten‑WT, Pumpen, Ventile betroffen.
- Maßnahmen: VDI 2035‑Wasser; Trinkwasserfilter; regelmäßige Checks; Spül-/Bypassöffnungen.

## Förderung/Timing

- Risiko: Doppelförderung, falsche Reihenfolge.
- Maßnahmen: EEE‑Begleitung; BzA vor Auftrag; saubere Kostentrennung; BnD zum Abschluss; Doku pflegen.

## Sicherheit (Ofen)

- Risiko: Überhitzung ohne Wärmesenke; zu kalter Rücklauf (Teer); unzureichender Ausdehnungsraum.
- Maßnahmen: RLA (≥60 °C), TAS, Notkühlweg, korrekt dimensioniertes MAG; zertifizierte Komponenten/Einbau.
\end{markdown}

\chapter{Offene Entscheidungen}
\begin{markdown}
# Offene Entscheidungen und Optionen

Zur Abstimmung mit Installationsbetrieb und EEE.

## TWW und Speicher

- Volumen TWW‑Heizspeicher: 200 L vs. 300 L (Nutzungsprofil/Parallellasten).
- FriWa‑Leistungsklasse: 25 kW vs. 35 kW (gleichzeitige Zapfungen).
- Option Querladung: Notfall‑Wärmeübertrag vom Heiz-/Kühlspeicher‑Top nach TWW (Komplexität vs. Resilienz).

## Entkopplung und Medien

- Optionaler PHE WP↔Haus (Glykol WP‑seitig):
  - Pro: Frostschutz, Sauerstoffeintrag begrenzt.
  - Contra: geringe Effizienzeinbuße, mehr Komponenten.

## Loggia/Eingang

- Loggia: Wintergarten (höchster Nutzen, Genehmigung) vs. zielgerichtete Innendämmung + Verschattung (Budget).
- Eingang: verglaster Windfang (hoher Nutzen) vs. reine Unterseitendämmung (geringerer Nutzen).

## WRG‑Geräte

- Typ: Doppellüfter‑Dauerbetrieb vs. Paar‑Wechsel; Akustikpriorisierung.
- Steuerung: Stand‑alone vs. Gateway in HA (Monitoring).

## Wärmepumpe

- Leistungsklasse: nach EN 12831 und Modulationsbereich.
- Akustik/Standort: Nachtmodus, Schallschirm, Nachbarschaft.

## Monitoring/Daten

- WMZ‑Umfang: nur WP vs. WP + FriWa‑Primär.
- Datenhaltung: erste Saison obligatorisch; optional Langzeittrends.
\end{markdown}

\chapter{Hydraulikschema — Tags und I/O}
\begin{markdown}
# Hydraulikschema — Kennzeichnung und I/O‑Plan

Ergänzende Kennzeichen (Tags) für Komponenten, Sensoren und Regel‑I/O zur Verdrahtung, Beschriftung und Inbetriebnahme.

## Tag‑Konvention

- Speicher: T1 = Heiz-/Kühlspeicher, T2 = TWW‑Heizspeicher
- Wärmepumpe: HP1
- Plattenwärmetauscher (optional): HX1 (WP↔Haus)
- Frischwasserstation: FW1
- Ofenkreis: WS1
- Pumpen: P‑xx, Ventile: V‑xx, Sensoren: S‑xx, Regler/Relais: C‑xx/R‑xx

## Komponenten/Tags

- HP1: Reversible L/W‑WP (R290)
- HX1: Optionaler PHE, Glykol WP‑seitig
- T1: Heiz-/Kühlspeicher (800–1000 L)
  - S‑T1‑TOP/MID/BOT (Temperaturen)
- T2: TWW‑Heizspeicher (200–300 L)
  - S‑T2‑TOP/MID/BOT (Temperaturen)
- FW1: FriWa (Platten‑WT, Primärpumpe, Auslauftemp‑Regelung)
  - S‑FW‑FLOW (Durchfluss), S‑FW‑OUT (TWW‑Auslauf)
- WS1: Wasserführender Ofen inkl. Sicherheit
  - V‑WS‑RL (Rücklaufanhebung ≥60 °C), V‑WS‑TD (thermische Ablaufsicherung)
- UFH: Verteiler/Schleifen
  - V‑MX‑UFH (3‑Wege‑Mischer), P‑UFH, S‑UFH‑VL/RL, S‑UFH‑SURF (optional)
- Saisonarmaturen
  - V‑SEAS‑T1/T2 für Service/Modus

## Raum‑Sensorik

- S‑AMB‑GF: EG T/r.F. (Taupunkt)
- S‑AMB‑DG: DG T/r.F. (Taupunkt)

## Elektro/Zähler

- R‑HP‑EN: WP‑Freigabe‑Relais
- M‑HP‑EL: Stromzähler WP
- M‑HP‑HT: WMZ WP→T1
- M‑FW‑HT: WMZ T2→FW1 (optional)

## I/O‑Plan (Beispiel)

- Regler C‑HYD (Hydraulik):
  - Eingänge: S‑T1‑TOP/MID/BOT, S‑T2‑TOP/MID/BOT, S‑UFH‑VL/RL, S‑AMB‑GF/DG, S‑UFH‑SURF (opt.)
  - Ausgänge: V‑MX‑UFH (0–10 V), P‑UFH (Ein/Aus oder PWM), P‑FW‑PRI (via FW1), R‑HP‑EN, V‑SEAS‑T1/T2, Alarm
  - Logik: Taupunktlimit; TWW‑Priorität; Saison; Antitakt; sicherer Stopp
- Regler C‑HP (in HP1):
  - Heizkurve; Vor-/Rücklaufgrenzen; Abtauung; Schnittstelle R‑HP‑EN
- FW1 intern:
  - Auslauftemperatur‑Soll; moduliert Primärpumpe nach Durchfluss/ΔT

## Beschriftung/Doku

- Jeder Tag erscheint in Schema, Verdrahtung, Geräteschild, Protokollen.
- Tag‑Legende ausdrucken und nahe T1/T2 anbringen.
\end{markdown}

\chapter{Betriebsleitfaden}
\begin{markdown}
# Betriebsleitfaden — Alltag und Saisontipps

Kurzanleitung für den täglichen Betrieb und saisonale Besonderheiten.

## Alltag

- TWW über FriWa im Durchlauf; konstante Auslauftemperatur. Bei parallelen Großzapfungen kurzzeitige Absenkung möglich; Speicher lädt nach.
- TWW‑Ladung bevorzugt mittags mit PV. Bei Schlechtwetter bleibt TWW‑Komfort dennoch gewährleistet.
- HA zeigt Puffertemperaturen, Luftfeuchte, Grundstatus; Meldungen für Filterservice/Abweichungen nutzen.

## Winter (Heizen)

- Wettergeführte Heizkurve mit niedrigen VL (~28–35 °C). Strahlungswärme baut sich träge, aber behaglich auf.
- Holzofen nach Wunsch: hebt Puffertop an, WP reduziert Leistung.
- Bei zu kühlen Räumen: Raum‑Soll leicht erhöhen oder Heizkurve minimal anpassen; große Sprünge vermeiden.

## Sommer (Kühlen)

- Nacht: Puffer ~16–18 °C; Tag: UFH‑VL durch Taupunktlogik begrenzt (typ. 19–21 °C); sanfte Grundkühlung.
- Steigt r.F. gegen 60 % und Kühlung pausiert: lüften bei trockener Außenluft oder kurzzeitig entfeuchten (kein AC‑Betrieb vorgesehen).

## Übergangszeit

- Niedrige Puffersollwerte; Komfort überwiegend passiv + kurze Heizphasen; TWW weiterhin mittags.

## Wartung

- Filter: WRG und FriWa Trinkwasserfilter alle 3–6 Monate sichten (Umgebungsabhängig).
- Sichtkontrolle: zu Sommerbeginn auf Kondensat an Kaltleitungen/Verteilern achten; Taupunkt‑Abstand ggf. erhöhen.
- Jährlich: Sicherheitsventile, MAG‑Vordruck, VDI 2035‑Wasser, Wärmemengenzähler prüfen.

## Störungshilfen

- TWW zu kühl: TWW‑Speicher, FriWa‑Soll, Trinkwasserfilter prüfen.
- Kühlung schwach: Taupunktlimit vs. VL checken; bei hoher r.F. Luft trocknen (Lüften/Entfeuchter).
- Geräusche: Nachtmodus (WP/WRG); Montage/Schalldämpfung prüfen.
\end{markdown}

\chapter{Kosten — Größenordnungen}
\begin{markdown}
# Kosten — Größenordnungen

Richtwerte zur Orientierung; verbindlich sind Angebote und aktuelle Förderbedingungen.

## Hülle

- Kellerdeckendämmung: 20–90 €/m²
- Dach/Oberste Decke: 50–200 €/m²
- Dreifachfenster inkl. luftdichter Montage: 600–1.000 € je Fenster (oder 280–900 €/m² Fensterfläche)
- WDVS Fassade: 90–210 €/m² (falls machbar)
- Luftdichtheit + Blower‑Door (vor/nach): 1.500–4.200 €
- Treppenhaus‑Abtrennung (Glas): 1.000–3.000 €
- Loggia Wintergarten: 5.000–20.000 € (abhängig vom Aufbau)
- Verglaster Windfang: ähnlich, objektabhängig

## Erzeugung/Speicher/Verteilung

- Reversible L/W‑WP (R290), installiert: 18.000–35.000 €
- Heiz-/Kühlspeicher 800–1000 L: 2.200–6.000 €
- TWW‑Heizspeicher 200–300 L: 800–1.800 €
- FriWa‑Station: 1.700–3.500 €
- UFH‑Nachrüstung: 60–145 €/m²
- Hydraulikperipherie + vollständige Rohrdämmung: 1.500–3.500 €
- Optionaler PHE (WP↔Haus): 800–2.700 €
- Wasserführender Ofen + Sicherheit/Abgasanlage: 3.500–10.000 € (+800–3.000 € Kaminanpassungen)

## Lüftung, PV, Elektro

- Dezentrale WRG (6–8 Geräte): 6.000–12.000 €
- PV 5,8–8 kWp: 8.000–12.000 € (0 % MwSt.) + Netzgebühren
- Batterie ~10 kWh mit Ersatzstrom: 6.000–10.000 € (0 % MwSt.)
- Unterverteilung wichtige Stromkreise + Umschaltung: 1.500–3.000 €

## Nebenkosten

- EEE (Bestätigungen, Baubegleitung): 2.000–5.000 €
- Technische Planung (Heiz/Kühllast, Schemata, LV, Bauüberwachung): 3.000–8.000 €
- Inbetriebnahme/Abgleich‑Protokolle: 1.000–2.500 €

## Summen (grob)

- Ohne Fassade: ~67.000–90.000 €
- Mit Fassade: ~82.000–115.000 €

Förderungen senken Nettokosten (z. B. EH‑Tilgungszuschuss; Einzelmaßnahmen 15–20 % mit iSFP; PV/Batterie 0 % MwSt.). Doppelförderung vermeiden; aktuelle Bedingungen vor Beauftragung prüfen.
\end{markdown}

\end{document}
