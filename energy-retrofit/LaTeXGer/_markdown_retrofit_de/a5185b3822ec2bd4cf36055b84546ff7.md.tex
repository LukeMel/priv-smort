\markdownRendererDocumentBegin
\markdownRendererSectionBegin
\markdownRendererHeadingOne{Projektüberblick}\markdownRendererInterblockSeparator
{}\markdownRendererUlBeginTight
\markdownRendererUlItem Gebäude: Reihenmittelhaus (~113 m²), Baujahr ~1972, EG + DG + Keller.\markdownRendererUlItemEnd 
\markdownRendererUlItem Belegung: 3 Personen.\markdownRendererUlItemEnd 
\markdownRendererUlItem Energie (2023):\markdownRendererUlItemEnd 
\markdownRendererUlItem Raumwärme: ~9.824 kWh/a (~87 kWh/m²·a), Fernwärme.\markdownRendererUlItemEnd 
\markdownRendererUlItem TWW‑Volumen: ~16,86 m³ (≈881 kWh/a thermisch, ΔT≈45 K).\markdownRendererUlItemEnd 
\markdownRendererUlItem Strom: ~2.671 kWh/a.\markdownRendererUlItemEnd 
\markdownRendererUlItem Ziele:\markdownRendererUlItemEnd 
\markdownRendererUlItem Effizienzhaus 85 + Erneuerbare‑Energien‑Klasse (≥65 - Hohen PV‑Eigenverbrauch, Resilienz durch Batteriespeicher und Unterverteilung für wichtige Stromkreise.\markdownRendererUlItemEnd 
\markdownRendererUlItem Einfache, effiziente Bedienung, sichere Regelung, saubere Inbetriebnahme.\markdownRendererUlItemEnd 
\markdownRendererUlItem Sanftes hydronisches Kühlen ohne Klimageräte (UFH + Taupunkt‑Schutz).\markdownRendererUlItemEnd 
\markdownRendererUlEndTight \markdownRendererInterblockSeparator
{}\markdownRendererSectionBegin
\markdownRendererHeadingTwo{Schlüsselentscheidungen}\markdownRendererInterblockSeparator
{}\markdownRendererUlBeginTight
\markdownRendererUlItem Doppelpuffer‑System:\markdownRendererUlItemEnd 
\markdownRendererUlItem TWW‑Heizspeicher (200–300 L) ganzjährig heiß (55–60 °C), speist FriWa (kein Trinkwasserspeicher).\markdownRendererUlItemEnd 
\markdownRendererUlItem Heiz-/Kühlspeicher (800–1.000 L) im Winter warm, im Sommer kalt (Ziel 16–18 °C für Nachtladung).\markdownRendererUlItemEnd 
\markdownRendererUlItem Wärmeerzeuger: Reversible Luft‑Wasser‑Wärmepumpe (L/W‑WP, R290), monovalent nach Heizlast; wasserführender Holzofen als Komfort/Backup.\markdownRendererUlItemEnd 
\markdownRendererUlItem Verteilung: Wassergeführte Fußbodenheizung (UFH) in Hauptzonen; keine Gebläsekonvektoren, keine AC.\markdownRendererUlItemEnd 
\markdownRendererUlItem Lüftung: Dezentrale Einzelraumgeräte (6–8 Stück) gemäß DIN 1946‑6.\markdownRendererUlItemEnd 
\markdownRendererUlItem Steuerung: Harte Sicherheits‑/Regeltechnik + Taupunktlogik; Home Assistant (HA) zur PV‑Orchestrierung/Monitoring.\markdownRendererUlItemEnd 
\markdownRendererUlEndTight \markdownRendererInterblockSeparator
{}
\markdownRendererSectionEnd \markdownRendererSectionBegin
\markdownRendererHeadingTwo{Zielwerte}\markdownRendererInterblockSeparator
{}\markdownRendererUlBeginTight
\markdownRendererUlItem Hülle (typisch EH85):\markdownRendererUlItemEnd 
\markdownRendererUlItem Kellerdecke U ≤ 0,25 W/(m²·K)\markdownRendererUlItemEnd 
\markdownRendererUlItem Dach/Oberste Decke U ≤ 0,14 W/(m²·K)\markdownRendererUlItemEnd 
\markdownRendererUlItem Fenster Uw ≤ 0,90 W/(m²·K)\markdownRendererUlItemEnd 
\markdownRendererUlItem Luftdichtheit n50 ≤ 1,5 h⁻¹ (Blower‑Door vorher/nachher)\markdownRendererUlItemEnd 
\markdownRendererUlItem Hydronik:\markdownRendererUlItemEnd 
\markdownRendererUlItem Heizen: Auslegung VL ≤ 35 °C; lange Verdichterlaufzeiten.\markdownRendererUlItemEnd 
\markdownRendererUlItem Kühlen: VL ≥ Taupunkt + 2 K, typ. 19–21 °C.\markdownRendererUlItemEnd 
\markdownRendererUlItem UFH‑Kühlleistung: ~10–25 W/m² (≈1,1–2,8 kW gesamt).\markdownRendererUlItemEnd 
\markdownRendererUlEndTight \markdownRendererInterblockSeparator
{}
\markdownRendererSectionEnd \markdownRendererSectionBegin
\markdownRendererHeadingTwo{Erwartete Wirkungen (grobe Ordnung)}\markdownRendererInterblockSeparator
{}\markdownRendererUlBeginTight
\markdownRendererUlItem Hülle + Wärmebrücken: ~25–40 - PV ~5,8–8 kWp: ~5,5–8,5 MWh/a; mit ~10 kWh Speicher 50–70 - Hydronisches Kühlen: angenehme Grundkühlung; Feuchte über Lüftung/Taupunktlogik geführt (bei Extremwetter ggf. mobiler Entfeuchter).\markdownRendererUlItemEnd 
\markdownRendererUlEndTight \markdownRendererInterblockSeparator
{}
\markdownRendererSectionEnd \markdownRendererSectionBegin
\markdownRendererHeadingTwo{Randbedingungen}\markdownRendererInterblockSeparator
{}\markdownRendererUlBeginTight
\markdownRendererUlItem Offenes Kellertreppenhaus: Luftdichte, transparente Abtrennung + Kellerdeckendämmung.\markdownRendererUlItemEnd 
\markdownRendererUlItem Loggia/Eingangsbereich: ausgeprägte Wärmebrücken; Optionen siehe 05_envelope-airtightness.md.\markdownRendererUlItemEnd 
\markdownRendererUlItem Keine zentrale Lüftung möglich: Einzelraumgeräte mit akustischer und strömungstechnischer Planung.\markdownRendererUlItemEnd 
\markdownRendererUlEndTight 
\markdownRendererSectionEnd 
\markdownRendererSectionEnd \markdownRendererDocumentEnd