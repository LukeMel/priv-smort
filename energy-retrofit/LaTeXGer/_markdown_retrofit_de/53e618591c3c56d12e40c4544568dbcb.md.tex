\markdownRendererDocumentBegin
\markdownRendererSectionBegin
\markdownRendererHeadingOne{Projektphasen und Reihenfolge}\markdownRendererInterblockSeparator
{}Pragmatische Abfolge zur Minimierung von Rückbau, passgenauer Förderung und qualitativem Hochlauf.\markdownRendererInterblockSeparator
{}\markdownRendererSectionBegin
\markdownRendererHeadingTwo{Phasen}\markdownRendererInterblockSeparator
{}1) Voruntersuchungen - Gefahrstoffe (Bj. 1972): Asbest/PCB/alte Mineralwolle. - Statik Flachdach für PV; Aufstellort/Schall; Genehmigungen (Loggia/Windfang).\markdownRendererInterblockSeparator
{}2) Konzept/Berechnungen - EN 12831 Heiz‑ (und ggf. Kühl‑)last; DIN 1946‑6 Lüftungskonzept. - Wärmebrückendetails Loggia/Eingang; Taupunktstrategie UFH. - Kostenzuordnung EH‑Topf vs. Einzelmaßnahmen.\markdownRendererInterblockSeparator
{}3) Förderanträge - BzA durch EEE; Bankgespräch. - KfW 261 (EH); ergänzende BAFA/BEG‑Einzelmaßnahmen und evtl. Zusatzkredite.\markdownRendererInterblockSeparator
{}4) Hülle/Luftdichtheit - Kellerdecke, Dach, Fenster; optional Fassade. - Treppenhaus‑Abtrennung; Loggia/Windfang (falls genehmigt). - Blower‑Door (Zwischenmessung) als QS.\markdownRendererInterblockSeparator
{}5) Technik - UFH‑Verteilung; Abgleich‑Bereitschaft. - WP, Speicher (TWW + Heiz/Kühl), FriWa, Pumpen/Ventile, Sensorik, Unterverteilung. - Dezentrale WRG inkl. Einregulierung.\markdownRendererInterblockSeparator
{}6) PV/Batterie - PV/Batterie; Unterverteilung integrieren; SPD.\markdownRendererInterblockSeparator
{}7) Inbetriebnahme/Feinabstimmung - Spülung/Druckprobe/VDI 2035; Parametrierung; Taupunkt‑Test; Ofen‑Sicherheitstests. - Abgleich‑Protokoll; Datenlogging.\markdownRendererInterblockSeparator
{}8) Abschluss - BnD (EEE); Zuschüsse/Tilgungszuschuss. - Erste Saison: Monitoring/Optimierung; ggf. Thermografie.
\markdownRendererSectionEnd 
\markdownRendererSectionEnd \markdownRendererDocumentEnd