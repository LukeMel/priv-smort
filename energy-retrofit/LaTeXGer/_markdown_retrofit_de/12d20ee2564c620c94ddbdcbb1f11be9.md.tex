\markdownRendererDocumentBegin
\markdownRendererSectionBegin
\markdownRendererHeadingOne{Regelung, Sensorik, Monitoring}\markdownRendererInterblockSeparator
{}Sicherheits‑ und Primärregelung sind hart verdrahtet; Home Assistant (HA) dient nur zur Orchestrierung/Visualisierung. Taupunkt‑Schutz ist Pflicht.\markdownRendererInterblockSeparator
{}\markdownRendererSectionBegin
\markdownRendererHeadingTwo{Sicherheit/Primärregelung (hart)}\markdownRendererInterblockSeparator
{}\markdownRendererUlBeginTight
\markdownRendererUlItem WP‑Regler: Verdichterschutz, Abtauung, Vor-/Rücklauftemperaturlimits.\markdownRendererUlItemEnd 
\markdownRendererUlItem Hydraulikregler: 3‑Wege‑Mischer UFH, Pumpen, Puffertemperaturgrenzen, TWW‑Priorität.\markdownRendererUlItemEnd 
\markdownRendererUlItem Taupunktabschaltung: harte Begrenzung UFH‑VL ≥ (Taupunkt + 2 K), unabhängig von HA.\markdownRendererUlItemEnd 
\markdownRendererUlItem Ofen: Rücklaufanhebung (≥ 60 °C), thermische Ablaufsicherung, schwerkraftgeeignete Notkühlung, MAG/PRV.\markdownRendererUlItemEnd 
\markdownRendererUlEndTight \markdownRendererInterblockSeparator
{}
\markdownRendererSectionEnd \markdownRendererSectionBegin
\markdownRendererHeadingTwo{Sensorik/Messtechnik}\markdownRendererInterblockSeparator
{}\markdownRendererUlBeginTight
\markdownRendererUlItem Speicher: oben/Mitte/unten an TWW- und Heiz-/Kühlspeicher.\markdownRendererUlItemEnd 
\markdownRendererUlItem UFH: VL/RL; optional Oberflächensensor am Verteiler.\markdownRendererUlItemEnd 
\markdownRendererUlItem Raum: mind. ein T/r.F. je Etage (Taupunktberechnung).\markdownRendererUlItemEnd 
\markdownRendererUlItem Energiemessung: Wärmemengen WP→Puffer, optional FriWa‑Primär; Strom‑Unterzähler WP.\markdownRendererUlItemEnd 
\markdownRendererUlItem Durchfluss/Druck: nach Bedarf (FriWa/Strangregulierung), Entlüftungen/Spülstellen.\markdownRendererUlItemEnd 
\markdownRendererUlEndTight \markdownRendererInterblockSeparator
{}
\markdownRendererSectionEnd \markdownRendererSectionBegin
\markdownRendererHeadingTwo{Orchestrierung (HA o. ä.)}\markdownRendererInterblockSeparator
{}\markdownRendererUlBeginTight
\markdownRendererUlItem PV‑Zeitfenster: TWW‑Ladung mittags; im Winter optional Puffer‑Vorwärmung bei PV‑Überschuss.\markdownRendererUlItemEnd 
\markdownRendererUlItem Saisons: Winter (Heizen), Sommer (Kühlen + TWW), Übergang (minimal).\markdownRendererUlItemEnd 
\markdownRendererUlItem Meldungen: Ofen‑Hinweise, Filterservice (WRG/FriWa), r.F./T‑Alarme, Fehlerrelais WP.\markdownRendererUlItemEnd 
\markdownRendererUlItem Logging: Puffer‑T, UFH VL/RL, r.F./T innen, WP‑Leistung, WMZ; Leistungszahl (COP)‑Proxy möglich.\markdownRendererUlItemEnd 
\markdownRendererUlEndTight \markdownRendererInterblockSeparator
{}
\markdownRendererSectionEnd \markdownRendererSectionBegin
\markdownRendererHeadingTwo{Beispiel Taupunktlogik}\markdownRendererInterblockSeparator
{}\markdownRendererUlBeginTight
\markdownRendererUlItem Eingang: T innen, r.F. innen → Taupunkt; gemessener UFH‑VL.\markdownRendererUlItemEnd 
\markdownRendererUlItem Limit: UFH‑VL‑Soll = max(Heizkurvenbedarf, Taupunkt + 2 K), absolut min. typ. 19–21 °C.\markdownRendererUlItemEnd 
\markdownRendererUlItem Sperre: Bei VL < (Taupunkt + 2 K) oder Feuchte am Verteiler: Mischer schließen/Pumpe stoppen bis sicher.\markdownRendererUlItemEnd 
\markdownRendererUlEndTight \markdownRendererInterblockSeparator
{}
\markdownRendererSectionEnd \markdownRendererSectionBegin
\markdownRendererHeadingTwo{Elektro/Resilienz}\markdownRendererInterblockSeparator
{}\markdownRendererUlBeginTight
\markdownRendererUlItem Unterverteilung wichtige Stromkreise: WP‑Freigabe, Pumpen, Hydraulikregler, HA‑Host, Netzwerk, Kühlgerät, ausgewählte Licht/Steckdosen.\markdownRendererUlItemEnd 
\markdownRendererUlItem Überspannungsschutz (SPD) für PV, Speicher, WP, MSR.\markdownRendererUlItemEnd 
\markdownRendererUlItem Nachtmodus: reduzierte Schallemission WP/WRG.\markdownRendererUlItemEnd 
\markdownRendererUlEndTight 
\markdownRendererSectionEnd 
\markdownRendererSectionEnd \markdownRendererDocumentEnd