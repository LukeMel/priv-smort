\markdownRendererDocumentBegin
\markdownRendererSectionBegin
\markdownRendererHeadingOne{Betriebsleitfaden — Alltag und Saisontipps}\markdownRendererInterblockSeparator
{}Kurzanleitung für den täglichen Betrieb und saisonale Besonderheiten.\markdownRendererInterblockSeparator
{}\markdownRendererSectionBegin
\markdownRendererHeadingTwo{Alltag}\markdownRendererInterblockSeparator
{}\markdownRendererUlBeginTight
\markdownRendererUlItem TWW über FriWa im Durchlauf; konstante Auslauftemperatur. Bei parallelen Großzapfungen kurzzeitige Absenkung möglich; Speicher lädt nach.\markdownRendererUlItemEnd 
\markdownRendererUlItem TWW‑Ladung bevorzugt mittags mit PV. Bei Schlechtwetter bleibt TWW‑Komfort dennoch gewährleistet.\markdownRendererUlItemEnd 
\markdownRendererUlItem HA zeigt Puffertemperaturen, Luftfeuchte, Grundstatus; Meldungen für Filterservice/Abweichungen nutzen.\markdownRendererUlItemEnd 
\markdownRendererUlEndTight \markdownRendererInterblockSeparator
{}
\markdownRendererSectionEnd \markdownRendererSectionBegin
\markdownRendererHeadingTwo{Winter (Heizen)}\markdownRendererInterblockSeparator
{}\markdownRendererUlBeginTight
\markdownRendererUlItem Wettergeführte Heizkurve mit niedrigen VL (~28–35 °C). Strahlungswärme baut sich träge, aber behaglich auf.\markdownRendererUlItemEnd 
\markdownRendererUlItem Holzofen nach Wunsch: hebt Puffertop an, WP reduziert Leistung.\markdownRendererUlItemEnd 
\markdownRendererUlItem Bei zu kühlen Räumen: Raum‑Soll leicht erhöhen oder Heizkurve minimal anpassen; große Sprünge vermeiden.\markdownRendererUlItemEnd 
\markdownRendererUlEndTight \markdownRendererInterblockSeparator
{}
\markdownRendererSectionEnd \markdownRendererSectionBegin
\markdownRendererHeadingTwo{Sommer (Kühlen)}\markdownRendererInterblockSeparator
{}\markdownRendererUlBeginTight
\markdownRendererUlItem Nacht: Puffer ~16–18 °C; Tag: UFH‑VL durch Taupunktlogik begrenzt (typ. 19–21 °C); sanfte Grundkühlung.\markdownRendererUlItemEnd 
\markdownRendererUlItem Steigt r.F. gegen 60 \markdownRendererUlItemEnd 
\markdownRendererUlEndTight \markdownRendererInterblockSeparator
{}
\markdownRendererSectionEnd \markdownRendererSectionBegin
\markdownRendererHeadingTwo{Übergangszeit}\markdownRendererInterblockSeparator
{}\markdownRendererUlBeginTight
\markdownRendererUlItem Niedrige Puffersollwerte; Komfort überwiegend passiv + kurze Heizphasen; TWW weiterhin mittags.\markdownRendererUlItemEnd 
\markdownRendererUlEndTight \markdownRendererInterblockSeparator
{}
\markdownRendererSectionEnd \markdownRendererSectionBegin
\markdownRendererHeadingTwo{Wartung}\markdownRendererInterblockSeparator
{}\markdownRendererUlBeginTight
\markdownRendererUlItem Filter: WRG und FriWa Trinkwasserfilter alle 3–6 Monate sichten (Umgebungsabhängig).\markdownRendererUlItemEnd 
\markdownRendererUlItem Sichtkontrolle: zu Sommerbeginn auf Kondensat an Kaltleitungen/Verteilern achten; Taupunkt‑Abstand ggf. erhöhen.\markdownRendererUlItemEnd 
\markdownRendererUlItem Jährlich: Sicherheitsventile, MAG‑Vordruck, VDI 2035‑Wasser, Wärmemengenzähler prüfen.\markdownRendererUlItemEnd 
\markdownRendererUlEndTight \markdownRendererInterblockSeparator
{}
\markdownRendererSectionEnd \markdownRendererSectionBegin
\markdownRendererHeadingTwo{Störungshilfen}\markdownRendererInterblockSeparator
{}\markdownRendererUlBeginTight
\markdownRendererUlItem TWW zu kühl: TWW‑Speicher, FriWa‑Soll, Trinkwasserfilter prüfen.\markdownRendererUlItemEnd 
\markdownRendererUlItem Kühlung schwach: Taupunktlimit vs. VL checken; bei hoher r.F. Luft trocknen (Lüften/Entfeuchter).\markdownRendererUlItemEnd 
\markdownRendererUlItem Geräusche: Nachtmodus (WP/WRG); Montage/Schalldämpfung prüfen.\markdownRendererUlItemEnd 
\markdownRendererUlEndTight 
\markdownRendererSectionEnd 
\markdownRendererSectionEnd \markdownRendererDocumentEnd