\markdownRendererDocumentBegin
\markdownRendererSectionBegin
\markdownRendererHeadingOne{Lüftung — Dezentrale Einzelraum‑WRG (DIN 1946‑6)}\markdownRendererInterblockSeparator
{}Da keine zentrale Lüftung möglich ist, kommen 6–8 dezentrale Geräte zum Einsatz. Sie stellen den Grundluftwechsel mit Wärmerückgewinnung sicher, besitzen Boost‑Funktionen und Sommerbypass.\markdownRendererInterblockSeparator
{}\markdownRendererSectionBegin
\markdownRendererHeadingTwo{Prinzipien}\markdownRendererInterblockSeparator
{}\markdownRendererUlBeginTight
\markdownRendererUlItem Bilanz/Abdeckung: Kontinuierlicher Grundluftwechsel in Wohn‑/Schlafräumen; stärkere Abfuhr/Boost in Feuchträumen.\markdownRendererUlItemEnd 
\markdownRendererUlItem Akustik: Leise im Grundbetrieb, Nachtmodus; akustische Einbauteile.\markdownRendererUlItemEnd 
\markdownRendererUlItem Einfacher Einbau: Kernbohrung 160–200 mm, leichtes Gefälle nach außen (Kondensat), Filterzugang.\markdownRendererUlItemEnd 
\markdownRendererUlEndTight \markdownRendererInterblockSeparator
{}
\markdownRendererSectionEnd \markdownRendererSectionBegin
\markdownRendererHeadingTwo{Gerätetypen}\markdownRendererInterblockSeparator
{}\markdownRendererUlBeginTight
\markdownRendererUlItem Doppellüfter‑Dauerbetrieb (bevorzugt): gleichzeitige Zu‑/Abluft über kleinen Gegenstromkern; stabile Luftbilanz.\markdownRendererUlItemEnd 
\markdownRendererUlItem Wechselbetrieb (Paarweise): Keramikspeicher; paarweise Montage zur Bilanzierung.\markdownRendererUlItemEnd 
\markdownRendererUlEndTight \markdownRendererInterblockSeparator
{}
\markdownRendererSectionEnd \markdownRendererSectionBegin
\markdownRendererHeadingTwo{Platzierung/Luftwege}\markdownRendererInterblockSeparator
{}\markdownRendererUlBeginTight
\markdownRendererUlItem Feuchträume (Bad/WC/Küche): Geräte mit Boost 40–60 m³/h; ggf. Fettfilter.\markdownRendererUlItemEnd 
\markdownRendererUlItem Wohn‑/Schlafräume: Grundvolumen 15–30 m³/h je Raum; Schlafräume mit Nachtmodus.\markdownRendererUlItemEnd 
\markdownRendererUlItem Türunterkanten/Überströmöffnungen; Kurzschlussströmung vermeiden.\markdownRendererUlItemEnd 
\markdownRendererUlItem Außenhauben mit Wetterschutz/Insektenschutz; Fassadenakustik beachten.\markdownRendererUlItemEnd 
\markdownRendererUlEndTight \markdownRendererInterblockSeparator
{}
\markdownRendererSectionEnd \markdownRendererSectionBegin
\markdownRendererHeadingTwo{Steuerung}\markdownRendererInterblockSeparator
{}\markdownRendererUlBeginTight
\markdownRendererUlItem Lokal: Taster für Boost (Bad), automatische Boosts (r.F./CO₂), Nachtmodi.\markdownRendererUlItemEnd 
\markdownRendererUlItem Zentrale Übersicht (optional): potentialfreie Kontakte oder Modbus/IP‑Gateway zur HA‑Integration (Monitoring, nicht sicherheitskritisch).\markdownRendererUlItemEnd 
\markdownRendererUlItem Sommerbypass: aktivieren, um unerwünschte Wärmerückgewinnung in der Kühlperiode zu vermeiden; Querlüftung nutzen.\markdownRendererUlItemEnd 
\markdownRendererUlEndTight \markdownRendererInterblockSeparator
{}
\markdownRendererSectionEnd \markdownRendererSectionBegin
\markdownRendererHeadingTwo{Einbau/Inbetriebnahme}\markdownRendererInterblockSeparator
{}\markdownRendererUlBeginTight
\markdownRendererUlItem Kernbohrungen mit leichtem Gefälle nach außen; luftdichte Hüll‑Anbindung.\markdownRendererUlItemEnd 
\markdownRendererUlItem Volumenströme gemäß DIN 1946‑6 einstellen, messen, protokollieren.\markdownRendererUlItemEnd 
\markdownRendererUlItem Filter: Wartungsintervalle definieren; Ersatzfiltersätze vorhalten; Service‑Hinweise in HA.\markdownRendererUlItemEnd 
\markdownRendererUlEndTight \markdownRendererInterblockSeparator
{}
\markdownRendererSectionEnd \markdownRendererSectionBegin
\markdownRendererHeadingTwo{Zusammenspiel mit hydronischem Kühlen}\markdownRendererInterblockSeparator
{}\markdownRendererUlBegin
\markdownRendererUlItem WRG hilft bei Feuchtestabilisierung, entfeuchtet aber nur begrenzt; r.F. pro Etage überwachen.\markdownRendererUlItemEnd 
\markdownRendererUlEnd \markdownRendererInterblockSeparator
{}- Bei anhaltend >60
\markdownRendererSectionEnd 
\markdownRendererSectionEnd \markdownRendererDocumentEnd