\markdownRendererDocumentBegin
\markdownRendererSectionBegin
\markdownRendererHeadingOne{Förderung, Konformität, Dokumentation}\markdownRendererInterblockSeparator
{}Strategie zur Förderung (Stand: konzeptionell). Verbindliche Prüfung stets mit Energie‑Effizienz‑Experte (EEE) und Bank vor Beauftragung.\markdownRendererInterblockSeparator
{}\markdownRendererSectionBegin
\markdownRendererHeadingTwo{Primär — KfW 261 (Wohngebäude – Kredit)}\markdownRendererInterblockSeparator
{}\markdownRendererUlBeginTight
\markdownRendererUlItem Ziel: Effizienzhaus 85 + Erneuerbare‑Energien‑Klasse (≥65 - Inhalt: Wärmepumpe, Speicher (Heiz/Kühl + TWW), FriWa, Hydraulik‑Peripherie, Sicherheit, System‑Elektro; ggf. wasserführender Ofen.\markdownRendererUlItemEnd 
\markdownRendererUlItem Tilgungszuschuss: typ. 10 - Wichtig: Keine Doppelförderung derselben Kostenposition außerhalb des EH‑Topfs.\markdownRendererUlItemEnd 
\markdownRendererUlEndTight \markdownRendererInterblockSeparator
{}
\markdownRendererSectionEnd \markdownRendererSectionBegin
\markdownRendererHeadingTwo{Ergänzend — Einzelmaßnahmen (BAFA/BEG)}\markdownRendererInterblockSeparator
{}\markdownRendererUlBeginTight
\markdownRendererUlItem Hülle (Keller/Dach/Fenster/Fassade), dezentrale Lüftung, Mess‑/Steuer‑/Regelung (MSR), UFH‑Verteilung.\markdownRendererUlItemEnd 
\markdownRendererUlItem Sätze: 15 - Regel: Keine Doppelansetzung, wenn bereits im EH‑Kredit enthalten.\markdownRendererUlItemEnd 
\markdownRendererUlEndTight \markdownRendererInterblockSeparator
{}
\markdownRendererSectionEnd \markdownRendererSectionBegin
\markdownRendererHeadingTwo{Weitere Finanzierung}\markdownRendererInterblockSeparator
{}\markdownRendererUlBeginTight
\markdownRendererUlItem Ergänzungskredite (z. B. KfW 358/359) für Einzelmaßnahmen.\markdownRendererUlItemEnd 
\markdownRendererUlItem PV/Batterie: 0 \markdownRendererUlItemEnd 
\markdownRendererUlEndTight \markdownRendererInterblockSeparator
{}
\markdownRendererSectionEnd \markdownRendererSectionBegin
\markdownRendererHeadingTwo{Prozess/Timing}\markdownRendererInterblockSeparator
{}\markdownRendererUlBeginTight
\markdownRendererUlItem EEE frühzeitig einbinden; „Bestätigung vor Antrag“ (BzA) vor Auftragsvergabe bzw. mit aufschiebender Bedingung.\markdownRendererUlItemEnd 
\markdownRendererUlItem Saubere Kostentrennung: EH‑Topf vs. Einzelmaßnahmen definieren.\markdownRendererUlItemEnd 
\markdownRendererUlItem „Bestätigung nach Durchführung“ (BnD) zur Zuschussgutschrift/Abschluss.\markdownRendererUlItemEnd 
\markdownRendererUlEndTight \markdownRendererInterblockSeparator
{}
\markdownRendererSectionEnd \markdownRendererSectionBegin
\markdownRendererHeadingTwo{Nachweise/Doku}\markdownRendererInterblockSeparator
{}\markdownRendererUlBeginTight
\markdownRendererUlItem Berechnungen: EN 12831 (Heizlast), Lüftungskonzept nach DIN 1946‑6, Taupunktstrategie, Wärmebrückendetails.\markdownRendererUlItemEnd 
\markdownRendererUlItem Inbetriebnahmeprotokolle: Blower‑Door (vor/nach), hydraulischer Abgleich, Wasserqualität (VDI 2035), Druckprobe, Wärmemengenzähler, WP‑Inbetriebnahme.\markdownRendererUlItemEnd 
\markdownRendererUlItem Sicherheit: Ofen EN 303‑5 (RLA, TAS, Notkühlung), Elektro‑SPDs, Unterverteilung für Notbetrieb.\markdownRendererUlItemEnd 
\markdownRendererUlItem Monitoring: erste Betriebsperiode (T, r.F., WMZ/EMZ) zur Performancevalidierung.\markdownRendererUlItemEnd 
\markdownRendererUlEndTight \markdownRendererInterblockSeparator
{}
\markdownRendererSectionEnd \markdownRendererSectionBegin
\markdownRendererHeadingTwo{Hinweise}\markdownRendererInterblockSeparator
{}\markdownRendererUlBeginTight
\markdownRendererUlItem Programme ändern sich; Konditionen/Caps/Fristen aktuell prüfen.\markdownRendererUlItemEnd 
\markdownRendererUlItem Eine „Single Source of Truth“ für Kostenzuordnung/Belege vereinfacht Prüfungen.\markdownRendererUlItemEnd 
\markdownRendererUlEndTight 
\markdownRendererSectionEnd 
\markdownRendererSectionEnd \markdownRendererDocumentEnd