\markdownRendererDocumentBegin
\markdownRendererSectionBegin
\markdownRendererHeadingOne{Trinkwarmwasser (TWW) — FriWa + TWW‑Heizspeicher}\markdownRendererInterblockSeparator
{}Separater TWW‑Heizspeicher (Heizungswasser) mit ~55–60 °C versorgt eine Frischwasserstation (FriWa). Es gibt keinen Trinkwasserspeicher; TWW wird im Durchlauf erwärmt.\markdownRendererInterblockSeparator
{}\markdownRendererSectionBegin
\markdownRendererHeadingTwo{Begründung}\markdownRendererInterblockSeparator
{}\markdownRendererUlBeginTight
\markdownRendererUlItem Hygiene: Kein stehendes Trinkwarmwasser → minimiertes Legionellenrisiko; Verbrühschutz an Zapfstellen.\markdownRendererUlItemEnd 
\markdownRendererUlItem Sommerkompatibilität: TWW‑Speicher bleibt heiß, Heiz-/Kühlspeicher kann kalt bleiben (UFH‑Kühlung).\markdownRendererUlItemEnd 
\markdownRendererUlItem PV‑Synergie: TWW‑Ladung mittags nutzt PV‑Überschuss, reduziert Abendlast.\markdownRendererUlItemEnd 
\markdownRendererUlEndTight \markdownRendererInterblockSeparator
{}
\markdownRendererSectionEnd \markdownRendererSectionBegin
\markdownRendererHeadingTwo{Hydraulik}\markdownRendererInterblockSeparator
{}\markdownRendererUlBeginTight
\markdownRendererUlItem TWW‑Heizspeicher (200–300 L): 55–60 °C ganzjährig; oben/Mitte/unten Sensoren; gute Dämmung.\markdownRendererUlItemEnd 
\markdownRendererUlItem FriWa (25–35 kW): Platten‑WT, variable Primärpumpe, Durchflusssensor, elektronische Auslauftemperatur; Trinkwasserseite mit Filter/Absperrungen.\markdownRendererUlItemEnd 
\markdownRendererUlItem WP‑Priorität: Zeitbasiert (mittags) und temperaturgeführt (wenn Speichertemperatur < Soll) zuerst TWW, sonst Heiz-/Kühlspeicher.\markdownRendererUlItemEnd 
\markdownRendererUlEndTight \markdownRendererInterblockSeparator
{}
\markdownRendererSectionEnd \markdownRendererSectionBegin
\markdownRendererHeadingTwo{Regelung/Sollwerte}\markdownRendererInterblockSeparator
{}\markdownRendererUlBeginTight
\markdownRendererUlItem TWW‑Soll: 55–60 °C (Wasserhärte/Komfort beachten). Verbrühschutzmischer einsetzen.\markdownRendererUlItemEnd 
\markdownRendererUlItem PV‑Fenster: Bei hohem SoC und PV‑Überschuss TWW laden; sonst zeitversetzt/off‑peak.\markdownRendererUlItemEnd 
\markdownRendererUlItem Stillstandsverluste: Speicher gut dämmen; ggf. Nachtabschaltung, wenn Bedarfsmuster es zulässt.\markdownRendererUlItemEnd 
\markdownRendererUlEndTight \markdownRendererInterblockSeparator
{}
\markdownRendererSectionEnd \markdownRendererSectionBegin
\markdownRendererHeadingTwo{Dimensionierung/Leistung}\markdownRendererInterblockSeparator
{}\markdownRendererUlBeginTight
\markdownRendererUlItem Speicher: 200–300 L für 3 Personen (Duschen, gelegentlich Baden); eher 300 L bei parallelen Zapfungen.\markdownRendererUlItemEnd 
\markdownRendererUlItem FriWa: 25–35 kW liefern typ. 10–16 L/min bei 40–45 °C (abhängig von Primärtemperatur/Kennlinien).\markdownRendererUlItemEnd 
\markdownRendererUlItem Annäherung ΔT: Primärtemperatur mit Reserve zur Auslauftemperatur; FriWa‑Regelung fein abstimmen.\markdownRendererUlItemEnd 
\markdownRendererUlEndTight \markdownRendererInterblockSeparator
{}
\markdownRendererSectionEnd \markdownRendererSectionBegin
\markdownRendererHeadingTwo{Wasserqualität/Wartung}\markdownRendererInterblockSeparator
{}\markdownRendererUlBeginTight
\markdownRendererUlItem Trinkwasserseite: Filter/Sieb; Inspektion/Entkalkung gemäß Wasserhärte.\markdownRendererUlItemEnd 
\markdownRendererUlItem Primärseite: VDI 2035 (Leitfähigkeit/Härte) zum Schutz von WT, Pumpen, Ventilen.\markdownRendererUlItemEnd 
\markdownRendererUlItem Service: Absperrungen, Entleerungen, Messstellen vorsehen.\markdownRendererUlItemEnd 
\markdownRendererUlEndTight \markdownRendererInterblockSeparator
{}
\markdownRendererSectionEnd \markdownRendererSectionBegin
\markdownRendererHeadingTwo{Inbetriebnahme‑Checks}\markdownRendererInterblockSeparator
{}\markdownRendererUlBeginTight
\markdownRendererUlItem Stabiler TWW‑Auslauf 2–16 L/min ohne Überschwingen.\markdownRendererUlItemEnd 
\markdownRendererUlItem Verbrühschutz prüfen.\markdownRendererUlItemEnd 
\markdownRendererUlItem TWW‑Ladezeiten/Temperaturverläufe loggen und PV‑Timing verifizieren.\markdownRendererUlItemEnd 
\markdownRendererUlEndTight 
\markdownRendererSectionEnd 
\markdownRendererSectionEnd \markdownRendererDocumentEnd