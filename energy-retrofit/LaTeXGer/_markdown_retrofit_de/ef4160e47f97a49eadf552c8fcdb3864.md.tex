\markdownRendererDocumentBegin
\markdownRendererSectionBegin
\markdownRendererHeadingOne{Gebäudehülle, Luftdichtheit, Wärmebrücken}\markdownRendererInterblockSeparator
{}Maßnahmen für Dämmung, Luftdichtheit und die neuralgischen Wärmebrücken (Loggia, offener Eingangsbereich, offenes Kellertreppenhaus).\markdownRendererInterblockSeparator
{}\markdownRendererSectionBegin
\markdownRendererHeadingTwo{Zielwerte (EH85‑nah)}\markdownRendererInterblockSeparator
{}\markdownRendererUlBeginTight
\markdownRendererUlItem Kellerdecke U ≤ 0,25 W/(m²·K)\markdownRendererUlItemEnd 
\markdownRendererUlItem Dach/Oberste Decke U ≤ 0,14 W/(m²·K)\markdownRendererUlItemEnd 
\markdownRendererUlItem Fenster Uw ≤ 0,90 W/(m²·K) mit warmer Kante\markdownRendererUlItemEnd 
\markdownRendererUlItem Luftdichtheit n50 ≤ 1,5 h⁻¹ (Blower‑Door vorher/nachher)\markdownRendererUlItemEnd 
\markdownRendererUlEndTight \markdownRendererInterblockSeparator
{}
\markdownRendererSectionEnd \markdownRendererSectionBegin
\markdownRendererHeadingTwo{Maßnahmen}\markdownRendererInterblockSeparator
{}\markdownRendererUlBeginTight
\markdownRendererUlItem Kellerdecke (muss): Platten oder Einblasdämmung; Durchdringungen abdichten; Anschlussdetails durchgängig.\markdownRendererUlItemEnd 
\markdownRendererUlItem Dach/Oberste Decke (muss): Aufdachdämmung oder oberste Decke; luftdichte Ebene konsequent; Details an Durchdringungen.\markdownRendererUlItemEnd 
\markdownRendererUlItem Fenster (muss): Dreifachverglasung, luftdichte Montage mit Bändern/Kompribändern; Laibungsdämmung prüfen.\markdownRendererUlItemEnd 
\markdownRendererUlItem Fassade (optional, empfohlen): WDVS außen, falls möglich; sonst kapillaraktive Innendämmung (z. B. CaSi/Holzfaser) an kritischen Innenflächen; Feuchteführung planen (keine Dampfsperrfallen).\markdownRendererUlItemEnd 
\markdownRendererUlItem Luftdichtheitspaket (muss): Professionelle Abdichtung, Blower‑Door vor/nach Sanierung.\markdownRendererUlItemEnd 
\markdownRendererUlEndTight \markdownRendererInterblockSeparator
{}
\markdownRendererSectionEnd \markdownRendererSectionBegin
\markdownRendererHeadingTwo{Offenes Kellertreppenhaus}\markdownRendererInterblockSeparator
{}\markdownRendererUlBeginTight
\markdownRendererUlItem Luftdichte, transparente Abtrennung (Glas/Schiebetür) auf Keller‑ oder EG‑Ebene; Lichtbezug erhalten; Dichtprofile.\markdownRendererUlItemEnd 
\markdownRendererUlItem In Kombination mit Kellerdeckendämmung verringert Kamineffekt.\markdownRendererUlItemEnd 
\markdownRendererUlEndTight \markdownRendererInterblockSeparator
{}
\markdownRendererSectionEnd \markdownRendererSectionBegin
\markdownRendererHeadingTwo{Loggia (innenliegender Balkon)}\markdownRendererInterblockSeparator
{}\markdownRendererUlBeginTight
\markdownRendererUlItem Wintergarten: Dreifachglas, thermisch getrennte Rahmen, gedämmte Brüstungen/Laibungen, außenliegende Verschattung; beste Behaglichkeit; Genehmigung/Statik beachten.\markdownRendererUlItemEnd 
\markdownRendererUlItem Zielgerichtete Wärmebrücken‑Sanierung: kapillaraktive Innendämmung an Knoten; ggf. außen ergänzen; geringer, aber günstiger Effekt.\markdownRendererUlItemEnd 
\markdownRendererUlEndTight \markdownRendererInterblockSeparator
{}
\markdownRendererSectionEnd \markdownRendererSectionBegin
\markdownRendererHeadingTwo{Offener Eingang/überdeckter Vorbereich}\markdownRendererInterblockSeparator
{}\markdownRendererUlBeginTight
\markdownRendererUlItem Verglaster Windfang: gedämmte Rahmen und Außentür; großer Nutzen gegen Infiltration/Kälte; ggf. Genehmigung.\markdownRendererUlItemEnd 
\markdownRendererUlItem Unterseitige Dämmung: starre PIR/EPS‑Platten; Seitenanschlüsse luftdicht/gedämmt.\markdownRendererUlItemEnd 
\markdownRendererUlEndTight \markdownRendererInterblockSeparator
{}
\markdownRendererSectionEnd \markdownRendererSectionBegin
\markdownRendererHeadingTwo{Verifikation}\markdownRendererInterblockSeparator
{}\markdownRendererUlBeginTight
\markdownRendererUlItem Thermografie nach Fertigstellung zur Restwärmebrückensuche.\markdownRendererUlItemEnd 
\markdownRendererUlItem Blower‑Door (n50 ≤ 1,5 h⁻¹), Leckagenachweise.\markdownRendererUlItemEnd 
\markdownRendererUlEndTight 
\markdownRendererSectionEnd 
\markdownRendererSectionEnd \markdownRendererDocumentEnd