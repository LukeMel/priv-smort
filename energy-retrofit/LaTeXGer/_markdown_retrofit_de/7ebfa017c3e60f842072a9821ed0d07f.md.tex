\markdownRendererDocumentBegin
\markdownRendererSectionBegin
\markdownRendererHeadingOne{Risiken und Gegenmaßnahmen}\markdownRendererInterblockSeparator
{}Wesentliche technische/projektbezogene Risiken und praxistaugliche Lösungen.\markdownRendererInterblockSeparator
{}\markdownRendererSectionBegin
\markdownRendererHeadingTwo{Hydronisches Kühlen/Kondensation}\markdownRendererInterblockSeparator
{}\markdownRendererUlBeginTight
\markdownRendererUlItem Risiko: UFH‑Oberflächen/Verteiler unter Taupunkt → Tauwasser/Schäden.\markdownRendererUlItemEnd 
\markdownRendererUlItem Maßnahmen: harte Taupunkt‑Limitierung (Mischer), diffusionsdichte Kältedämmung, Kondensatführung, r.F.‑Überwachung je Etage, Kühlung bei >60 \markdownRendererUlItemEnd 
\markdownRendererUlEndTight \markdownRendererInterblockSeparator
{}
\markdownRendererSectionEnd \markdownRendererSectionBegin
\markdownRendererHeadingTwo{Systemkomplexität}\markdownRendererInterblockSeparator
{}\markdownRendererUlBeginTight
\markdownRendererUlItem Risiko: Doppelpuffer + FriWa + Ofen erhöhen Komponenten/Regelung.\markdownRendererUlItemEnd 
\markdownRendererUlItem Maßnahmen: Sicherheitsverriegelungen in dedizierten Reglern (nicht HA), klare Betriebsmodi, strukturierte Inbetriebnahme, Beschriftung, Servicedoku.\markdownRendererUlItemEnd 
\markdownRendererUlEndTight \markdownRendererInterblockSeparator
{}
\markdownRendererSectionEnd \markdownRendererSectionBegin
\markdownRendererHeadingTwo{Schall (Reihenhaus)}\markdownRendererInterblockSeparator
{}\markdownRendererUlBeginTight
\markdownRendererUlItem Risiko: Außengerät/WRG stören Bewohner/Nachbarn.\markdownRendererUlItemEnd 
\markdownRendererUlItem Maßnahmen: akustisch günstiger Standort/Schirm, entkoppelte Montage, Nachtmodus, WRG‑Akustikmaßnahmen.\markdownRendererUlItemEnd 
\markdownRendererUlEndTight \markdownRendererInterblockSeparator
{}
\markdownRendererSectionEnd \markdownRendererSectionBegin
\markdownRendererHeadingTwo{Wärmebrücken (Loggia, Eingang)}\markdownRendererInterblockSeparator
{}\markdownRendererUlBeginTight
\markdownRendererUlItem Risiko: Energieverlust, kalte Oberflächen, Feuchte.\markdownRendererUlItemEnd 
\markdownRendererUlItem Maßnahmen: bevorzugt Wintergarten/Windfang; alternativ kapillaraktive Innendämmung + außenliegende Verschattung; Thermografie nach Fertigstellung.\markdownRendererUlItemEnd 
\markdownRendererUlEndTight \markdownRendererInterblockSeparator
{}
\markdownRendererSectionEnd \markdownRendererSectionBegin
\markdownRendererHeadingTwo{Wasserqualität/Verkalkung}\markdownRendererInterblockSeparator
{}\markdownRendererUlBeginTight
\markdownRendererUlItem Risiko: Platten‑WT, Pumpen, Ventile betroffen.\markdownRendererUlItemEnd 
\markdownRendererUlItem Maßnahmen: VDI 2035‑Wasser; Trinkwasserfilter; regelmäßige Checks; Spül-/Bypassöffnungen.\markdownRendererUlItemEnd 
\markdownRendererUlEndTight \markdownRendererInterblockSeparator
{}
\markdownRendererSectionEnd \markdownRendererSectionBegin
\markdownRendererHeadingTwo{Förderung/Timing}\markdownRendererInterblockSeparator
{}\markdownRendererUlBeginTight
\markdownRendererUlItem Risiko: Doppelförderung, falsche Reihenfolge.\markdownRendererUlItemEnd 
\markdownRendererUlItem Maßnahmen: EEE‑Begleitung; BzA vor Auftrag; saubere Kostentrennung; BnD zum Abschluss; Doku pflegen.\markdownRendererUlItemEnd 
\markdownRendererUlEndTight \markdownRendererInterblockSeparator
{}
\markdownRendererSectionEnd \markdownRendererSectionBegin
\markdownRendererHeadingTwo{Sicherheit (Ofen)}\markdownRendererInterblockSeparator
{}\markdownRendererUlBeginTight
\markdownRendererUlItem Risiko: Überhitzung ohne Wärmesenke; zu kalter Rücklauf (Teer); unzureichender Ausdehnungsraum.\markdownRendererUlItemEnd 
\markdownRendererUlItem Maßnahmen: RLA (≥60 °C), TAS, Notkühlweg, korrekt dimensioniertes MAG; zertifizierte Komponenten/Einbau.\markdownRendererUlItemEnd 
\markdownRendererUlEndTight 
\markdownRendererSectionEnd 
\markdownRendererSectionEnd \markdownRendererDocumentEnd