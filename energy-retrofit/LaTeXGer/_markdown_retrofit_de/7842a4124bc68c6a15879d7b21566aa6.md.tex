\markdownRendererDocumentBegin
\markdownRendererSectionBegin
\markdownRendererHeadingOne{Inbetriebnahme und Abnahme}\markdownRendererInterblockSeparator
{}Checkliste für sichere Inbetriebnahme und förderkonforme Nachweise.\markdownRendererInterblockSeparator
{}\markdownRendererSectionBegin
\markdownRendererHeadingTwo{Vor‑Inbetriebnahme}\markdownRendererInterblockSeparator
{}\markdownRendererUlBeginTight
\markdownRendererUlItem Mechanik:\markdownRendererUlItemEnd 
\markdownRendererUlItem Spülung/Druckprobe bestanden; dicht.\markdownRendererUlItemEnd 
\markdownRendererUlItem MAG dimensioniert und voreingestellt (Gesamtvolumen Speicher+Anlage).\markdownRendererUlItemEnd 
\markdownRendererUlItem Luft/Schlammabscheider gesetzt; Entlüftungen zugänglich.\markdownRendererUlItemEnd 
\markdownRendererUlItem Dämmung vollständig: warm gedämmt; kalt diffusionsdicht inkl. Kondensatführung.\markdownRendererUlItemEnd 
\markdownRendererUlItem Elektro:\markdownRendererUlItemEnd 
\markdownRendererUlItem Unterverteilung wichtige Stromkreise verdrahtet; Beschriftung; RCD/LS geprüft.\markdownRendererUlItemEnd 
\markdownRendererUlItem SPDs installiert und mit PV/Batterie koordiniert.\markdownRendererUlItemEnd 
\markdownRendererUlItem Sensorik/Aktoren beschriftet; Not‑Stop dokumentiert.\markdownRendererUlItemEnd 
\markdownRendererUlItem Wasserqualität:\markdownRendererUlItemEnd 
\markdownRendererUlItem VDI‑Richtlinie 2035 (Verein Deutscher Ingenieure) konformes Wasser; Leitfähigkeit/Härte protokolliert.\markdownRendererUlItemEnd 
\markdownRendererUlItem Sicherheit (Ofen):\markdownRendererUlItemEnd 
\markdownRendererUlItem Rücklaufanhebung, thermische Ablaufsicherung, Notkühlweg, Schornsteinfreigaben geprüft.\markdownRendererUlItemEnd 
\markdownRendererUlEndTight \markdownRendererInterblockSeparator
{}
\markdownRendererSectionEnd \markdownRendererSectionBegin
\markdownRendererHeadingTwo{Regelung/Sensorik prüfen}\markdownRendererInterblockSeparator
{}\markdownRendererUlBeginTight
\markdownRendererUlItem Speicher: oben/Mitte/unten plausibel; Flussrichtungen geprüft.\markdownRendererUlItemEnd 
\markdownRendererUlItem UFH: VL/RL‑Sensoren, 3‑Wege‑Mischer‑Richtung, Pumpendrehzahl.\markdownRendererUlItemEnd 
\markdownRendererUlItem Taupunktlogik: Feuchte‑Szenario testen; Begrenzung/Abschaltung verifizieren.\markdownRendererUlItemEnd 
\markdownRendererUlItem WP‑Regler: Heizkurve/Min‑Max VL/Antitakt eingestellt.\markdownRendererUlItemEnd 
\markdownRendererUlItem FriWa: Auslauftemperatur stabil, 2–16 L/min; Verbrühschutz verifiziert.\markdownRendererUlItemEnd 
\markdownRendererUlEndTight \markdownRendererInterblockSeparator
{}
\markdownRendererSectionEnd \markdownRendererSectionBegin
\markdownRendererHeadingTwo{Funktionstests}\markdownRendererInterblockSeparator
{}\markdownRendererUlBeginTight
\markdownRendererUlItem Heizen (Winter‑Simulation):\markdownRendererUlItemEnd 
\markdownRendererUlItem Lange Verdichterläufe; Schichtung sichtbar; Ziel‑VL erreicht.\markdownRendererUlItemEnd 
\markdownRendererUlItem TWW‑Mittagspriorität auf 55–60 °C; danach Heizen.\markdownRendererUlItemEnd 
\markdownRendererUlItem Kühlen (Sommer‑Simulation):\markdownRendererUlItemEnd 
\markdownRendererUlItem Nachtkühlung Puffer ~16–18 °C; Tag VL = Taupunkt + 2 K; keine Kondensation.\markdownRendererUlItemEnd 
\markdownRendererUlItem Ofen‑Einbindung:\markdownRendererUlItemEnd 
\markdownRendererUlItem Puffer oben aufladen; WP‑Leistungsreduktion bei hoher Top‑Temperatur.\markdownRendererUlItemEnd 
\markdownRendererUlEndTight \markdownRendererInterblockSeparator
{}
\markdownRendererSectionEnd \markdownRendererSectionBegin
\markdownRendererHeadingTwo{Hydraulischer Abgleich}\markdownRendererInterblockSeparator
{}\markdownRendererUlBeginTight
\markdownRendererUlItem UFH‑Kreise messen, einstellen, protokollieren.\markdownRendererUlItemEnd 
\markdownRendererUlItem WMZ prüfen (Einbaurichtung, Impulse fürs Logging).\markdownRendererUlItemEnd 
\markdownRendererUlEndTight \markdownRendererInterblockSeparator
{}
\markdownRendererSectionEnd \markdownRendererSectionBegin
\markdownRendererHeadingTwo{Lüftung (Dezentrale WRG)}\markdownRendererInterblockSeparator
{}\markdownRendererUlBeginTight
\markdownRendererUlItem Raumweise Volumenströme einstellen; Boost; Sommerbypass.\markdownRendererUlItemEnd 
\markdownRendererUlItem Filter eingesetzt; Wartungsplan; Akustik prüfen.\markdownRendererUlItemEnd 
\markdownRendererUlEndTight \markdownRendererInterblockSeparator
{}
\markdownRendererSectionEnd \markdownRendererSectionBegin
\markdownRendererHeadingTwo{Abnahmedokumente}\markdownRendererInterblockSeparator
{}\markdownRendererUlBeginTight
\markdownRendererUlItem Blower‑Door (vor/nach), n50 ≤ 1,5 h⁻¹ (Ziel).\markdownRendererUlItemEnd 
\markdownRendererUlItem Protokolle hydraulischer Abgleich, WP/FriWa‑Inbetriebnahme.\markdownRendererUlItemEnd 
\markdownRendererUlItem VDI 2035, Druckprüfungen, Elektro (RCD/Isolationsmessung/SPD).\markdownRendererUlItemEnd 
\markdownRendererUlItem Schemata: finales Hydraulik‑Schema, I/O‑Plan, Sensorliste.\markdownRendererUlItemEnd 
\markdownRendererUlItem Sicherheit: EN 303‑5 Nachweise, Schornsteinfeger‑Abnahmen, Notfall‑Prozeduren.\markdownRendererUlItemEnd 
\markdownRendererUlItem Monitoringplan: zu erfassende Daten, Aufbewahrungsdauer (mind. 1. Saison).\markdownRendererUlItemEnd 
\markdownRendererUlEndTight 
\markdownRendererSectionEnd 
\markdownRendererSectionEnd \markdownRendererDocumentEnd