\markdownRendererDocumentBegin
\markdownRendererSectionBegin
\markdownRendererHeadingOne{Offene Entscheidungen und Optionen}\markdownRendererInterblockSeparator
{}Zur Abstimmung mit Installationsbetrieb und EEE.\markdownRendererInterblockSeparator
{}\markdownRendererSectionBegin
\markdownRendererHeadingTwo{TWW und Speicher}\markdownRendererInterblockSeparator
{}\markdownRendererUlBeginTight
\markdownRendererUlItem Volumen TWW‑Heizspeicher: 200 L vs. 300 L (Nutzungsprofil/Parallellasten).\markdownRendererUlItemEnd 
\markdownRendererUlItem FriWa‑Leistungsklasse: 25 kW vs. 35 kW (gleichzeitige Zapfungen).\markdownRendererUlItemEnd 
\markdownRendererUlItem Option Querladung: Notfall‑Wärmeübertrag vom Heiz-/Kühlspeicher‑Top nach TWW (Komplexität vs. Resilienz).\markdownRendererUlItemEnd 
\markdownRendererUlEndTight \markdownRendererInterblockSeparator
{}
\markdownRendererSectionEnd \markdownRendererSectionBegin
\markdownRendererHeadingTwo{Entkopplung und Medien}\markdownRendererInterblockSeparator
{}\markdownRendererUlBeginTight
\markdownRendererUlItem Optionaler PHE WP↔Haus (Glykol WP‑seitig):\markdownRendererUlItemEnd 
\markdownRendererUlItem Pro: Frostschutz, Sauerstoffeintrag begrenzt.\markdownRendererUlItemEnd 
\markdownRendererUlItem Contra: geringe Effizienzeinbuße, mehr Komponenten.\markdownRendererUlItemEnd 
\markdownRendererUlEndTight \markdownRendererInterblockSeparator
{}
\markdownRendererSectionEnd \markdownRendererSectionBegin
\markdownRendererHeadingTwo{Loggia/Eingang}\markdownRendererInterblockSeparator
{}\markdownRendererUlBeginTight
\markdownRendererUlItem Loggia: Wintergarten (höchster Nutzen, Genehmigung) vs. zielgerichtete Innendämmung + Verschattung (Budget).\markdownRendererUlItemEnd 
\markdownRendererUlItem Eingang: verglaster Windfang (hoher Nutzen) vs. reine Unterseitendämmung (geringerer Nutzen).\markdownRendererUlItemEnd 
\markdownRendererUlEndTight \markdownRendererInterblockSeparator
{}
\markdownRendererSectionEnd \markdownRendererSectionBegin
\markdownRendererHeadingTwo{WRG‑Geräte}\markdownRendererInterblockSeparator
{}\markdownRendererUlBeginTight
\markdownRendererUlItem Typ: Doppellüfter‑Dauerbetrieb vs. Paar‑Wechsel; Akustikpriorisierung.\markdownRendererUlItemEnd 
\markdownRendererUlItem Steuerung: Stand‑alone vs. Gateway in HA (Monitoring).\markdownRendererUlItemEnd 
\markdownRendererUlEndTight \markdownRendererInterblockSeparator
{}
\markdownRendererSectionEnd \markdownRendererSectionBegin
\markdownRendererHeadingTwo{Wärmepumpe}\markdownRendererInterblockSeparator
{}\markdownRendererUlBeginTight
\markdownRendererUlItem Leistungsklasse: nach EN 12831 und Modulationsbereich.\markdownRendererUlItemEnd 
\markdownRendererUlItem Akustik/Standort: Nachtmodus, Schallschirm, Nachbarschaft.\markdownRendererUlItemEnd 
\markdownRendererUlEndTight \markdownRendererInterblockSeparator
{}
\markdownRendererSectionEnd \markdownRendererSectionBegin
\markdownRendererHeadingTwo{Monitoring/Daten}\markdownRendererInterblockSeparator
{}\markdownRendererUlBeginTight
\markdownRendererUlItem WMZ‑Umfang: nur WP vs. WP + FriWa‑Primär.\markdownRendererUlItemEnd 
\markdownRendererUlItem Datenhaltung: erste Saison obligatorisch; optional Langzeittrends.\markdownRendererUlItemEnd 
\markdownRendererUlEndTight 
\markdownRendererSectionEnd 
\markdownRendererSectionEnd \markdownRendererDocumentEnd