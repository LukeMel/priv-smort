% Energy Retrofit Documentation (English) — Self-contained
\documentclass[11pt,oneside]{report}
\usepackage[utf8]{inputenc}
\usepackage[T1]{fontenc}
\usepackage{lmodern}
\usepackage{geometry}
\geometry{a4paper,margin=1in}
\usepackage[hidelinks]{hyperref}
\usepackage{bookmark}
\usepackage[english]{babel}
\usepackage{textcomp}
\usepackage{newunicodechar}
\usepackage[hybrid]{markdown}
% Unicode mapping for robust compilation
\newunicodechar{≥}{$\ge$}
\newunicodechar{≤}{$\le$}
\newunicodechar{°}{$^\circ$}
\newunicodechar{·}{\ensuremath{\cdot}}
\newunicodechar{−}{-}
\newunicodechar{–}{--}
\newunicodechar{—}{---}
\newunicodechar{↔}{$\leftrightarrow$}
\newunicodechar{→}{$\to$}
\newunicodechar{≈}{$\approx$}
\newunicodechar{²}{\ensuremath{^2}}
\newunicodechar{Δ}{$\Delta$}
\newunicodechar{₂}{$_2$}
\newunicodechar{⁻}{$^{-}$}
\newunicodechar{€}{\texteuro{}}

\title{Energy Retrofit Documentation\\Dual-Buffer Architecture, HRV, Controls, Funding}
\author{Project: 1972 Mid-Terrace House}
\date{\today}

\begin{document}
\maketitle
\setcounter{secnumdepth}{3}
\setcounter{tocdepth}{3}
\tableofcontents
\clearpage

\chapter{Abbreviations}
\begin{markdown}
- AWHP — air‑to‑water heat pump
- DHW — domestic hot water
- FriWa — fresh‑water station (instantaneous DHW via plate heat exchanger)
- UFH — underfloor heating
- HRV — heat recovery ventilation
- PHE — plate heat exchanger
- PV — photovoltaics
- HA — Home Assistant (open‑source automation)
- HP — heat pump
- COP — coefficient of performance
- SoC — state of charge (battery)
- SPD — surge protective device
- ATS — automatic transfer switch
- EEE — Energie‑Effizienz‑Experte (energy‑efficiency expert)
- EH — Efficiency House standard (e.g., EH 85)
- DIN — Deutsches Institut für Normung (German standards)
- EN — European Norm (standards)
- KfW — Kreditanstalt für Wiederaufbau (German development bank)
- BAFA — Bundesamt für Wirtschaft und Ausfuhrkontrolle
- BEG — Bundesförderung für effiziente Gebäude
- iSFP — individueller Sanierungsfahrplan (individual retrofit roadmap)
- BzA — Bestätigung zum Antrag (confirmation before application)
- BnD — Bestätigung nach Durchführung (confirmation after completion)
- VDI — Verein Deutscher Ingenieure (Association of German Engineers)
- WMZ — heat meter (Wärmemengenzähler)
- RCD — residual current device
- MCB — miniature circuit breaker
- RLA — return‑lift valve (keeps stove return ≥ ~60°C)
- TAS — thermal discharge safety (valve)
\end{markdown}

\chapter{Project Overview}
\begin{markdown}
# Project Overview

- Building: Mid‑terrace house (Reihenmittelhaus), ~113 m², built ~1972, EG + DG + cellar.
- Occupancy: 3 persons.
- Current energy (2023):
  - Space heat: ~9,824 kWh/yr (~87 kWh/m²·yr), district heating.
- Domestic hot water (DHW): ~16.86 m³ water (~≈881 kWh/yr thermal, ΔT≈45 K).
  - Electricity: ~2,671 kWh/yr.
- Goals:
  - Achieve Effizienzhaus 85 + Erneuerbare‑Energien‑Klasse (≥65% renewable share for heat/cold).
- Strong photovoltaics (PV) self‑consumption; resilience with battery and critical‑loads subpanel.
  - Efficient, simple‑to‑operate system with safe controls and clean commissioning.
  - Gentle hydronic cooling without AC (UFH cooling with dew‑point protection).

## Key Design Decisions

- Dual‑buffer system:
  - DHW heating buffer (200–300 L) always hot (55–60°C), supplies FriWa (no potable storage).
  - Heating/Cooling buffer (800–1,000 L) hot in winter, cold in summer (16–18°C target for night charging).
- Heat source: Reversible air‑to‑water heat pump (AWHP, R290), monovalent for design heat load; wood stove with back‑boiler as comfort/backup.
- Distribution: Water‑based underfloor heating (UFH) in main zones; no fan‑coils, no AC.
- Ventilation: Decentralized single‑room heat recovery ventilation (HRV) units (6–8 total), DIN 1946‑6 compliant.
- Controls: Hard‑wired safety + dew‑point logic; Home Assistant for PV‑aware orchestration and monitoring.

## Performance Targets

- Envelope targets (typical for EH 85):
  - Basement ceiling U ≤ 0.25 W/(m²·K)
  - Roof/attic U ≤ 0.14 W/(m²·K)
  - Windows Uw ≤ 0.90 W/(m²·K)
  - Airtightness n50 ≤ 1.5 h⁻¹ (pre/post blower‑door tests)
- Hydronic targets:
  - Heating design VL ≤ 35°C; long compressor runs.
  - Cooling VL ≥ dew point + 2 K, typically 19–21°C.
  - UFH cooling capacity expectation: ~10–25 W/m² (≈1.1–2.8 kW total).

## Expected Impacts (order‑of‑magnitude)

- Envelope + thermal bridge fixes: ~25–40% reduction in space‑heating energy vs. current baseline.
- PV ~5.8–8 kWp: ~5.5–8.5 MWh/yr generation; with battery 10 kWh, self‑consumption 50–70% typically achievable.
- Hydronic cooling: pleasant background cooling; humidity managed by ventilation and dew‑point limit (add dehumidifier only during unusual heat/humidity spells).

## Constraints and Notes

- Open basement staircase: add airtight glazed partition and insulate basement ceiling.
- Loggia (interior balcony) and open entry are thermal bridges; see 05_envelope‑airtightness.md for options (winter garden / vestibule prioritized).
- No central ventilation feasible; plan single‑room HRV units with proper placement and acoustics.
\end{markdown}

\chapter{Historical Consumption and Costs (2021--2023)}
\begin{markdown}
Figures extracted from district‑heating statements (see source files below). Focused on amounts and costs.

## Key Figures

- 2021
  - Space heat: 14,542 kWh — 2,158.76 EUR
  - DHW: 37.86 m³ — 341.11 EUR
  - Total billed (heat + DHW): 2,499.88 EUR
- 2022
  - Space heat: 10,898 kWh — 1,917.08 EUR
  - DHW: 23.00 m³ — 232.04 EUR
  - Total billed (heat + DHW): 2,149.12 EUR
- 2023
  - Space heat: 9,824 kWh — 3,197.43 EUR
  - DHW: 16.86 m³ — 445.41 EUR
  - Total billed (heat + DHW): 3,642.84 EUR
  - CO2 info (statement): factor 0.251 kg CO2/kWh; emissions 2,961 kg; CO2 cost 95.04 EUR

## Source Documents

- 2021: `infosAndStuff/2021_10000027695_818349.pdf`
- 2022: `infosAndStuff/2022_10000027695_818349.pdf`
- 2023: `infosAndStuff/2023_10000027695_818349.pdf`
\end{markdown}

\chapter{Design Architecture — Dual Buffers}
\begin{markdown}
# Design Architecture — Dual Buffers

This document defines the hydraulic concept, operating modes, key setpoints, and component sizing for the dual‑buffer plan: a dedicated always‑hot DHW heating buffer feeding a fresh‑water station (FriWa), plus a separate heating/cooling buffer that is hot in winter and cold in summer. No air conditioners or fan‑coils are used; cooling is via underfloor heating (UFH) with dew‑point protection.

## One‑Line Hydraulic Schematic (ASCII)

\end{markdown}
\begin{Verbatim}[fontsize=\small]
           [PV + Battery + Critical Loads Subpanel]
                          |
                   [Electrical Supply]
                          |
                       [AWHP R290]
                          |
                   (optional Plate HX)
                          |
                 +--------+---------+
                 |                  |
       [Heating/Cooling Buffer]   [DHW Heating Buffer]
          800-1000 L, strat.         200-300 L, 55-60 C
                 |                  |
                 |                  +--> [FriWa] --> Potable DHW
                 |                           ^
                 |                           |
           +-----+----+                 Cold mains in
           |          |
        3-way       Return
        mixing        from
        valve        UFH
           |          ^
           v          |
         [UFH Supply] +---> [UFH Loops] ---> [UFH Return]

 [Wood Stove w/Back-Boiler] --(return lift, safety)--> Buffer Top
\end{Verbatim}
\begin{markdown}

Notes:
- The DHW heating buffer contains closed system water (not potable). The FriWa heats potable water on demand via a plate heat exchanger.
- The heating/cooling buffer acts as hydraulic separator and energy store. In summer it is chilled at night; in winter it is kept warm.
- An optional plate heat exchanger (PHE) between the AWHP and indoor circuit allows glycol on the AWHP side and protects indoor water quality.

## Operating Modes

- Winter
  - AWHP runs weather‑compensated to charge the heating/cooling buffer (mid‑zone) for space heating; VL design ≤ 35°C.
  - Midday PV window: AWHP prioritizes DHW heating buffer to 55–60°C for FriWa draws; thereafter returns to space heating.
  - Wood stove charges the buffer top via return‑lift valve (≥60°C) and has thermal discharge safety and gravity emergency cooling per EN 303‑5.
  - Home Assistant (HA) orchestrates priorities and notifications; safety/limits are hard‑wired in dedicated controllers.

- Summer
  - Night: AWHP chills the heating/cooling buffer to ~16–18°C.
  - Day: UFH supply is limited to ≥(dew point + 2 K), typically 19–21°C, to avoid condensation.
  - DHW heating buffer remains hot year‑round; AWHP charges it around midday using PV.

- Shoulder Seasons
  - Minimal buffer setpoints; DHW midday priority; occasional night‑chill during heatwaves.

## Setpoints, Limits, and Sensors

- DHW heating buffer: 55–60°C setpoint; anti‑scald mixing valves at fixtures if needed.
- Heating/Cooling buffer (winter): weather‑compensated targets to achieve VL ≤ 35°C at design.
- Heating/Cooling buffer (summer): night target 16–18°C.
- UFH supply in cooling: ≥(dew point + 2 K). Example: 26°C and 60% RH → dew point ≈ 17.8°C → UFH supply ≥ 20°C.
- Sensors (minimum):
  - Buffers: top/mid/bottom temperatures on both tanks.
  - UFH: supply and return temperatures; optional surface sensor at manifold.
  - Ambient: at least one temperature/RH sensor per floor for dew‑point calculation.
  - Heat/cold metering: heat meters on AWHP to buffer and FriWa primary if feasible.

## Component Sizing (Guidance)

- AWHP (reversible, R290): select by EN 12831 heat loss; typical post‑retrofit terrace house 4–6 kW at −10°C; choose 6–10 kW class with good turndown and quiet operation.
- Heating/Cooling buffer: 800–1000 L, stratified, 100–150 mm insulation, low Δp connections, 3 thermowells (top/mid/bottom).
- DHW heating buffer: 200–300 L, 3 thermowells, maintained at 55–60°C year‑round.
- FriWa: 25–35 kW plate HX module with variable‑speed primary pump, flow sensor, electronic outlet control; potable‑side filtration/anti‑scale as required by local water hardness.
- Valves/Pumps: ECM circulation pumps; 3‑way mixing valve for UFH with hard dew‑point input; motorized isolation valves for summer/winter decoupling; check valves and balancing valves.
- Optional PHE: sized for full thermal capacity with low approach ΔT; glycol on AWHP side; include air/dirt separators on both circuits.
- Expansion vessels: sized for total volume (buffers + pipework + emitters); verify per manufacturer; often 50–80 L combined.

## Safety and Water Quality

- Stove loop: return‑lift (e.g., 60°C cartridge), thermal discharge safety valve to drain, gravity‑safe emergency cooling path, adequate expansion vessel (EN 303‑5 compliance).
- System protection: pressure relief valves (typically 3 bar), air/dirt separators, VDI 2035 water treatment (conductivity/hardness), full insulation of hot lines and diffusion‑tight insulation on all cold lines with managed condensate.
- Electrical: critical‑loads subpanel to maintain HP enable, pumps, controls, network, fridge, and selected lighting/outlets during outages via battery.

## Control Allocation

- Safety/primary control: heat pump native controller; dedicated hydronic controller for mixing valves, pumps, temperature limits, dew‑point cutout; hard interlocks, not dependent on HA.
- Orchestration (HA): PV‑aware scheduling (e.g., DHW midday), seasonal mode toggles, notifications (wood stove use, filter service, abnormal RH/temps), data logging (buffer temps, RH, energy meters, COP proxy).
\end{markdown}

\chapter{Domestic Hot Water (DHW)}
\begin{markdown}
# Domestic Hot Water (DHW) — FriWa + DHW Heating Buffer

This design uses a dedicated DHW heating buffer (closed system water) held at ~55–60°C, feeding a Frischwasserstation (FriWa). There is no potable water storage; potable water is heated on demand through a plate heat exchanger.

## Rationale

- Hygiene: No stored potable hot water → minimized legionella risk; simple anti‑scald control at outlets.
- Summer compatibility: The DHW buffer remains hot while the separate heating/cooling buffer can stay cold for UFH cooling.
- PV synergy: Midday charging of the DHW buffer with PV improves self‑consumption and reduces evening load.

## Hydraulics

- DHW heating buffer (200–300 L): maintained at 55–60°C year‑round; fitted with top/mid/bottom sensors and good insulation.
- FriWa module (25–35 kW): includes plate HX, primary variable‑speed pump, flow sensor, electronic temperature control; potable‑side filter/strainer and service valves.
- AWHP priority logic: DHW charging has time‑of‑day priority (midday) and temperature priority (if buffer top < setpoint); otherwise AWHP serves space heating/cooling buffer.

## Control and Setpoints

- DHW buffer setpoint: 55–60°C (tune to water hardness and comfort). Use anti‑scald mixing valves at fixtures.
- PV‑aware charging: If battery state of charge (SoC) is high and PV surplus available, run DHW charge to target; otherwise defer to off‑peak schedule.
- Standby losses: Keep buffer well insulated; consider night setback only if draw patterns allow and comfort unaffected.

## Sizing and Performance

- Buffer volume: 200–300 L is typically sufficient for 3 persons with showers and occasional bath; choose larger end if simultaneous draws are common.
- FriWa capacity: 25–35 kW HX module typically gives 10–16 L/min at 40–45°C with sufficient primary temperature; confirm with manufacturer curves.
- Approach temperature: Ensure primary (DHW buffer) temperature margin to meet outlet setpoint during peak draws; tune FriWa PID.

## Water Quality and Maintenance

- Potable side: inline filter/strainer; periodic inspection/descaling based on local hardness.
- Primary side (system water): VDI 2035 treatment to protect plate HX, pumps, valves; monitor conductivity/hardness.
- Serviceability: install isolation valves, drains, and test taps for FriWa and DHW buffer.

## Commissioning Checks

- Verify FriWa outlet stability from 2–16 L/min; ensure no overshoot/undershoot.
- Confirm anti‑scald protection at outlets (thermostatic mixers where applicable).
- Log DHW charging periods and temperatures; validate PV alignment.
\end{markdown}

\chapter{Ventilation — Decentralized HRV}
\begin{markdown}
# Ventilation — Decentralized Single‑Room HRV (DIN 1946‑6)

No central ventilation is feasible; this plan uses 6–8 decentralized HRV units to provide base air exchange with heat recovery, local boosts, and summer bypass.

## Design Principles

- Balance and coverage: Provide continuous background ventilation in living/bedrooms; stronger extraction or boost in wet rooms.
- Acoustic comfort: Low noise at base flow, night mode for bedrooms, acoustic liners/baffles.
- Simplicity: Short core drill per unit (160–200 mm), slight outward slope for condensate, easy filter access.

## Unit Types

- Dual‑fan continuous units (preferred): simultaneous supply and exhaust through a small counter‑flow core; more stable balance.
- Alternating push‑pull units (paired): ceramic core stores heat; install in pairs to approximate balance.

## Placement and Air Paths

- Wet rooms (bath, WC, kitchen): units with boost 40–60 m³/h; grease filters where needed.
- Living/bedrooms: base 15–30 m³/h per room; night mode in bedrooms.
- Door undercuts/transfer grilles to enable crossflow; avoid short‑circuiting supply to immediate exhaust.
- Exterior intakes with weather hoods and insect screens; consider façade acoustics.

## Controls

- Local control: manual boost switches (bath fans), humidity/CO₂‑based automatic boosts, night modes.
- Central overview (optional): dry contacts or Modbus/IP bridge to integrate with Home Assistant for visibility and logging (not safety‑critical).
- Summer bypass: enable to prevent unwanted heat recovery during cooling season; querventilation possible.

## Installation and Commissioning

- Core drills with slight outward slope for condensate; ensure airtight sleeve sealing.
- Configure design flow rates per DIN 1946‑6 ventilation concept; measure and record flows.
- Filters: set maintenance intervals; keep spare sets; include filter‑service notifications.

## Interaction with Hydronic Cooling

- HRV aids humidity control but does not dehumidify aggressively; monitor RH per floor.
- If RH > 60% persists during heat events, consider temporary standalone dehumidifier while maintaining “no AC” constraint.
\end{markdown}

\chapter{Envelope, Airtightness, Thermal Bridges}
\begin{markdown}
# Envelope, Airtightness, Thermal Bridges

This section defines measures for insulation, airtightness, and mitigation of key thermal bridges (loggia, open entrance/vestibule, open staircase to cellar).

## Targets (EH85‑aligned)

- Basement ceiling U ≤ 0.25 W/(m²·K)
- Roof/attic U ≤ 0.14 W/(m²·K)
- Windows Uw ≤ 0.90 W/(m²·K) with warm edge spacers
- Airtightness n50 ≤ 1.5 h⁻¹ (blower‑door before/after)

## Measures

- Basement ceiling insulation (must): rigid boards or blown‑in solutions; ensure continuous coverage around beams and services; seal penetrations.
- Roof/attic insulation (must): on‑roof insulation or upper ceiling; maintain continuous airtight layer; detail penetrations and edges.
- Windows (must): triple glazing, airtight installation with tapes and compressible seals; verify reveal insulation.
- Facade (optional, recommended): external WDVS where feasible; if not, capillary‑active internal insulation (e.g., CaSi/wood fibre) at critical interior surfaces; model details to avoid vapor traps.
- Airtightness package (must): professional sealing, tape schedules, and blower‑door test pre/post.

## Staircase to Cellar (Open Stair)

- Install airtight glazed partition/door at basement or ground‑floor landing; maintain natural light; perimeter seals.
- Combine with basement ceiling insulation to cut stack‑effect losses.

## Loggia (Interior Balcony) Options

- Winter garden conversion: high‑performance triple glazing, thermally broken frames, insulated parapets/soffits, exterior shading; best comfort and thermal‑bridge fix; requires structural and permit checks.
- Targeted thermal‑bridge remediation: capillary‑active internal insulation at junctions; add exterior insulation if feasible; less impact but lower cost.

## Open Entrance / Covered Entry (Below Living Space)

- Glazed vestibule (Windfang): insulated framing and external door; major infiltration/comfort improvement; may require permit.
- Underside insulation: rigid PIR/EPS below ceiling; seal side‑wall junctions; economical but smaller impact.

## Verification

- Thermography after completion to identify residual thermal bridges.
- Blower‑door test (n50 ≤ 1.5 h⁻¹) with documented leakage sealing actions.
\end{markdown}

\chapter{Controls, Sensors, Monitoring}
\begin{markdown}
# Controls, Sensors, Monitoring

Safety controls must be hard‑wired; Home Assistant (HA) is used for orchestration and monitoring only. Dew‑point protection is mandatory for UFH cooling.

## Safety/Primary Controls (Hard‑wired)

- Heat pump controller: compressor protections, defrost logic, supply/return limits.
- Hydronic controller: 3‑way mixing valve control for UFH, pump control, buffer temperature limits, DHW priority.
- Dew‑point cutout: direct input limiting UFH supply temperature to ≥(dew point + 2 K); independent of HA.
- Stove safety: return‑lift valve (≥60°C), thermal discharge safety valve, gravity emergency cooling path; pressure relief; expansion vessel sizing.

## Sensors and Instrumentation

- Buffers: top/mid/bottom temperatures on DHW and heating/cooling buffers.
- UFH: supply and return temperatures; optional manifold surface temperature.
- Ambient: at least one temperature/RH sensor per floor for dew point.
- Energy metering: heat meters for AWHP→buffer and FriWa primary (optional), electric sub‑meter for HP.
- Flow/pressure: as required by FriWa and UFH balancing; include purge points and drains.

## Orchestration (Home Assistant or similar)

- PV‑aware schedules: DHW buffer charge at midday; optional winter buffer preheat during PV surplus.
- Seasonal modes: winter (heating), summer (cooling + DHW only), shoulder (adaptive minimal setpoints).
- Notifications: wood stove opportunity prompts; filter service (HRV/FriWa); abnormal RH/temperature alerts; fault relays from HP.
- Data logging: buffer temps, UFH VL/RL, indoor RH/T, HP power, heat meter data; derive coefficient of performance (COP) proxy where meters available.

## Example Dew‑Point Logic

- Inputs: indoor T, indoor RH → dew point; measured UFH supply temperature.
- Limit: UFH supply target = max(heating curve demand, dew point + 2 K) with absolute min typically 19–21°C.
- Fault/lockout: if supply < (dew point + 2 K) or manifold surface moisture detected, close mixing valve/stop pump until conditions safe.

## Electrical and Resilience

- Critical‑loads subpanel: HP enable, circulators, hydronic controller, HA host, network gear, fridge, selected lights/outlets.
- Surge protection: SPD coordination for PV, battery, HP, control electronics.
- Night mode: reduce acoustic output of outdoor unit and HRV where possible.
\end{markdown}

\chapter{Funding, Compliance, Documentation}
\begin{markdown}
# Funding, Compliance, Documentation

This section outlines a funding approach aligned with current concepts. Always verify current program rules and eligibility with an Energie‑Effizienz‑Experte (EEE) and your bank before committing.

## Primary Route — KfW 261 (Wohngebäude – Kredit)

- Goal: Effizienzhaus 85 + Erneuerbare‑Energien‑Klasse (≥65% renewable share for heat/cold).
- Credit scope: heat pump system, buffers, FriWa, hydraulic periphery, safety, system electricals; potentially wood stove with back‑boiler as part of the system scope.
- Repayment bonus: typically 10% on the financed portion once confirmed by EEE after completion.
- Important: Costs financed within the EH loan pot cannot be double‑funded elsewhere.

## Complementary — Einzelmaßnahmen (BAFA/BEG)

- Apply to building envelope (basement/roof/windows/facade), decentralized HRV, measurement/control/automation (MSR), and UFH distribution.
- Typical rates: 15% or 20% with iSFP (individueller Sanierungsfahrplan).
- Rule: No double‑funding of the same cost position if already in KfW 261 EH pot.

## Additional Financing

- Supplementary loans (e.g., KfW programs 358/359) for Einzelmaßnahmen up to common caps per dwelling.
- PV and battery: 0% VAT on supply/install; can be financed separately (e.g., KfW 270) or via bank.

## Process and Timing

- Engage an EEE early. Obtain “Bestätigung zum Antrag” (BzA) before awarding contracts, or use suspensive clauses.
- Clean cost separation: define which positions are under EH‑loan vs. Einzelmaßnahmen to avoid double‑funding.
- Post‑completion “Bestätigung nach Durchführung” (BnD) to trigger repayment bonus and finalize grants.

## Documentation and Compliance

- Design calculations: EN 12831 heat load; ventilation concept per DIN 1946‑6; dew‑point logic; thermal bridge details.
- Commissioning protocols: blower‑door (pre/post), hydraulic balancing, water quality (VDI 2035), pressure tests, heat meter setup, HP commissioning.
- Safety: stove per EN 303‑5 with return lift, thermal discharge, emergency cooling; electrical SPDs and backup subpanel documentation.
- Monitoring: first‑season data logs (temperatures, humidity, heat/power meters) to validate performance.

## Notes

- Funding programs evolve. Validate all rates, caps, and eligibility windows at application time.
- Maintain a single source of truth for cost allocation and supporting documents to simplify audits.
\end{markdown}

\chapter{Commissioning and Acceptance}
\begin{markdown}
# Commissioning and Acceptance

This checklist defines the procedures and evidence required to safely commission and accept the system. It is structured to match funding documentation needs.

## Pre‑Commissioning

- Mechanical:
  - Pipework flushed, pressure tested; leak‑free.
  - Expansion vessels sized and pre‑charged for total volume (buffers + circuits).
  - Air/dirt separators installed at strategic points; vents accessible.
  - Insulation complete: hot lines insulated; cold lines diffusion‑tight with condensate routes.
- Electrical:
  - Critical‑loads subpanel wired; circuits labeled; RCD/MCB verified.
  - Surge protection devices (SPDs) installed and coordinated with PV/battery.
  - Control wiring (sensors, valves, pumps) labeled; emergency stops documented.
- Water quality:
  - System water prepared per VDI 2035; conductivity/hardness logged.
- Safety (stove):
  - Return‑lift valve operation verified; thermal discharge safety connected to drain; gravity emergency cooling path tested; chimney approvals.

## Controls and Sensors Validation

- Buffers: confirm top/mid/bottom sensor readings; directions of flow validated.
- UFH: verify supply/return sensors; 3‑way valve orientation; pump rotation.
- Dew‑point logic: inject high RH in test zone or simulate via controller; confirm UFH supply limit and lockout behavior.
- HP controller: weather curve loaded; limits set (min/max VL, anti‑short‑cycle).
- FriWa: outlet temperature control tuned; stable performance over 2–16 L/min draws; anti‑scald verified.

## Functional Tests

- Heating mode (winter simulation):
  - Long steady compressor operation; buffer stratification observed; target VL achieved at test outdoor setpoint.
  - DHW midday priority: buffer top reaches 55–60°C; resume space heating post‑charge.
- Cooling mode (summer simulation):
  - Night buffer chill to ~16–18°C; daytime UFH VL maintained at dew point + 2 K; no condensation on manifolds/lines.
- Stove integration:
  - Charge buffer top; verify AWHP priority reduction on high top‑of‑buffer temperature.

## Hydraulic Balancing

- UFH circuits: measure and record volume flows per loop; adjust to design; document setpoints.
- Heat meters: verify installation direction and pulse outputs if logging.

## Ventilation (Decentralized HRV)

- Design flows set per room; boost functions verified; summer bypass configured.
- Filters installed; maintenance calendar established; acoustic checks.

## Acceptance Documentation

- Blower‑door test results (pre + post) with achieved n50 ≤ 1.5 h⁻¹ (target).
- Hydronic balancing protocol (flows per loop), HP and FriWa commissioning reports.
- VDI 2035 water quality report; pressure test certificates; electrical test reports (RCD/insulation/SPD).
- Schematics: final hydraulic diagram, control I/O map, sensor list.
- Safety attestations: stove compliance (EN 303‑5), chimney sweep approvals, emergency procedures.
- Monitoring plan: which data is logged and retention period (first season minimum).
\end{markdown}

\chapter{Project Phases and Recommended Order}
\begin{markdown}
# Project Phases and Recommended Order

A practical sequence minimizes rework, aligns with funding steps, and protects commissioning quality.

## Phases

1) Pre‑checks and Surveys
- Hazard screening (1972 build): asbestos in adhesives/fibre‑cement, PCB, old mineral wool.
- Roof statics for PV; outdoor unit siting and acoustics; local permits (loggia/vestibule).

2) Concept and Calculations
- EN 12831 heat and (if needed) cooling loads; DIN 1946‑6 ventilation concept.
- Thermal‑bridge detailing at loggia and entry; dew‑point strategy for UFH cooling.
- Define cost allocation to EH loan vs. Einzelmaßnahmen.

3) Funding Applications
- EEE issues BzA (confirmation before application) and supports bank discussion.
- Submit KfW 261 EH loan; plan complementary Einzelmaßnahmen (BAFA/BEG) and any supplementary loans.

4) Envelope and Airtightness
- Basement ceiling, roof/attic, windows; optional facade.
- Stair partition; loggia/vestibule if approved.
- Blower‑door test (intermediate if feasible) for QA.

5) Technical Systems
- UFH installation and distribution; hydraulic balancing readiness.
- AWHP, buffers (DHW + heating/cooling), FriWa, pumps/valves, sensors, electrical subpanel.
- Decentralized HRV units with commissioning of design flows.

6) PV and Battery
- PV array and battery install; integrate critical‑loads subpanel; SPDs.

7) Commissioning and Tuning
- Flush, pressure test, VDI 2035; parameterization; dew‑point test; stove safety tests.
- Hydraulic balancing protocol; data logging setup.

8) Post‑Completion
- EEE issues BnD (confirmation after completion); repayment bonus/grants processed.
- First‑season monitoring; optimize setpoints; perform thermography if needed.
\end{markdown}

\chapter{Bill of Materials}
\begin{markdown}
# Bill of Materials (Specification Classes)

Brand‑agnostic list with size ranges. Final selections should follow detailed calcs and installer standards.

## Heat Generation and Storage

- Reversible AWHP (R290), 6–10 kW class, monobloc or split; low noise kit; night mode.
- Optional Plate Heat Exchanger (AWHP↔House), sized for full capacity with low approach ΔT; glycol on AWHP side.
- Heating/Cooling Buffer: 800–1000 L, stratified, 100–150 mm insulation, 3× thermowells.
- DHW Heating Buffer: 200–300 L, high insulation, 3× thermowells.
- FriWa Module: 25–35 kW plate HX, variable‑speed primary pump, flow sensor, electronic outlet control, service valves, potable filter/strainer.
- Wood Stove w/Back‑Boiler (optional): rated output matched to buffer; return‑lift valve (≥60°C), thermal discharge valve, emergency cooling path components; chimney parts as required.

## Hydronic Periphery

- Circulation Pumps: ECM pumps for AWHP loop(s), buffer charging, UFH circuits, FriWa primary.
- Valves: 3‑way mixing valve (UFH), motorized zone valves for seasonal decoupling, check valves, balancing valves, drain/fill valves.
- Separators: air and dirt separators at key locations; magnetic dirt separator if needed.
- Expansion Vessels: sized for total water volume; service valves and gauges.
- Safety: PRVs (typically 3 bar), manometers, automatic air vents; condensate traps for cold lines.

## Distribution

- UFH Manifolds and Loops: oxygen‑barrier PEX/MLCP, manifold cabinets, flow meters, actuators if zoned.
- Pipe Insulation: hot lines to code; cold lines diffusion‑tight; manifold/valve box insulation where possible.

## Ventilation (Decentralized HRV)

- 6–8 single‑room HRV units (dual‑fan continuous or push‑pull pairs), wall sleeves, exterior hoods, acoustic liners, filters.
- Control accessories: boost switches, humidity/CO₂ sensors (where supported), integration gateway (optional).

## Controls, Sensors, Electrical

- Hydronic Controller: mixing valve + pump control with dew‑point input and lockout.
- Sensors: buffer temps (top/mid/bottom), UFH VL/RL, ambient T/RH per floor, optional manifold surface probe.
- Energy Meters: heat meter(s) on AWHP and FriWa primary (optional), sub‑meter for HP electrical.
- Electrical: critical‑loads subpanel, automatic transfer switch (ATS) for backup, surge protective devices (SPDs), labeling, wiring accessories.
- Home Assistant Host: small, reliable compute (e.g., SBC or mini‑PC), network connectivity, UPS (optional).

## Water Treatment and Service

- VDI 2035 treatment unit/chemistry, test kit (conductivity/hardness), fill/drain assemblies.
- Filters/Strainers: potable filter for FriWa; strainers where needed on primary; spare filter sets.
- Serviceability: isolation valves, drain points, thermowells/test points, access panels.
\end{markdown}

\chapter{Risks and Mitigations}
\begin{markdown}
# Risks and Mitigations

Key technical and project risks with practical mitigations.

## Hydronic Cooling and Condensation

- Risk: UFH surface and manifolds falling below dew point → condensation, damage.
- Mitigations: hard dew‑point limit controlling mixing valve; diffusion‑tight insulation on cold lines; condensate routing; humidity monitoring per floor; disable cooling if RH persistently > 60%.

## System Complexity

- Risk: Dual buffers + FriWa + stove add components and controls.
- Mitigations: safety interlocks in dedicated controllers (not HA); clear operating modes; thorough commissioning; labeled valves/wiring; service documentation.

## Noise (Terrace Context)

- Risk: Outdoor unit and HRV noise disturbing occupants/neighbors.
- Mitigations: acoustic siting and shielding; resilient mounts; night mode; façade‑friendly HRV placements with baffles.

## Thermal Bridges (Loggia, Entry)

- Risk: Heat loss, cold surfaces, moisture.
- Mitigations: Prefer enclosure (winter garden/vestibule); otherwise capillary‑active internal insulation and exterior shading; thermography post‑works.

## Water Quality and Scaling

- Risk: Scaling and corrosion in plate HX, pumps, valves.
- Mitigations: VDI 2035 treatment; potable‑side filters; periodic checks; bypass/flush ports for service.

## Funding/Timing Errors

- Risk: Double‑funding, out‑of‑sequence contracting jeopardizing eligibility.
- Mitigations: EEE involvement; BzA before award; clean cost split; BnD after completion; maintain documentation.

## Safety (Stove Integration)

- Risk: Overheating without heat dump; low return temp tar formation; insufficient expansion.
- Mitigations: return‑lift valve (≥60°C), thermal discharge valve to drain, gravity emergency cooling path, expansion vessel sizing; certified components and installation.
\end{markdown}

\chapter{Open Decisions and Options}
\begin{markdown}
# Open Decisions and Options

Use this page to track pending choices and finalize with the installer/EEE.

## DHW and Buffers

- Confirm DHW heating buffer volume: 200 L vs. 300 L based on peak draw habits.
- FriWa capacity class: 25 kW vs. 35 kW based on simultaneous draw expectations.
- Optional cross‑charge: allow emergency heat transfer from heating/cooling buffer top to DHW buffer (complexity vs. resilience).

## Separation and Fluids

- Optional AWHP↔House plate heat exchanger (glycol on AWHP side):
  - Pros: freeze protection and oxygen ingress isolation.
  - Cons: small efficiency penalty; added components.

## Loggia and Entryway

- Loggia: winter garden enclosure (permit, cost, highest benefit) vs. targeted internal insulation + exterior shading (budget option).
- Entry: glazed vestibule (high impact) vs. underside insulation only (lower impact).

## HRV Units

- Device type: dual‑fan continuous vs. push‑pull pairs; acoustic priorities.
- Controls: standalone vs. optional gateway into HA for monitoring.

## Heat Pump Selection

- Capacity class: based on EN 12831 result and modulation range.
- Acoustic package and siting: night mode, shielding, neighbor impact.

## Monitoring and Data

- Heat meters scope: AWHP only vs. AWHP + FriWa primary.
- Data retention: first season mandatory; optional long‑term trends.
\end{markdown}

\chapter{Hydraulic Diagram — Tags and I/O Map}
\begin{markdown}
# Hydraulic Diagram — Tags and I/O Map

This augments the architecture with reference tags for components, sensors, and control I/O. Use it to drive wiring, labels, and commissioning.

## Tagging Convention

- Tanks: T1 = Heating/Cooling Buffer, T2 = DHW Heating Buffer
- Heat Pump: HP1
- Plate Heat Exchanger (optional): HX1 (AWHP↔House)
- FriWa: FW1
- Wood Stove Loop: WS1
- Pumps: P‑xx, Valves: V‑xx, Sensors: S‑xx, Controllers/Relays: C‑xx/R‑xx

## Components and Tags

- HP1: Reversible AWHP (R290)
- HX1: Optional PHE with glycol on HP1 side
- T1: Heating/Cooling Buffer (800–1000 L)
  - S‑T1‑TOP, S‑T1‑MID, S‑T1‑BOT (temperatures)
- T2: DHW Heating Buffer (200–300 L)
  - S‑T2‑TOP, S‑T2‑MID, S‑T2‑BOT (temperatures)
- FW1: FriWa module (plate HX, primary pump, outlet control)
  - S‑FW‑FLOW (flow sensor), S‑FW‑OUT (DHW outlet temp)
- WS1: Wood stove with back‑boiler and safety kit
  - V‑WS‑RL (return‑lift valve ≥60°C), V‑WS‑TD (thermal discharge safety)
- UFH: Manifold(s) and loops
  - V‑MX‑UFH (3‑way mixing valve), P‑UFH (circulator), S‑UFH‑VL/S‑UFH‑RL (temps), S‑UFH‑SURF (optional surface)
- Seasonal Decoupling Valves
  - V‑SEAS‑T1 (isolate T1), V‑SEAS‑T2 (isolate T2) as needed for service/mode control

## Ambient Sensors

- S‑AMB‑GF: Ground floor T/RH (dew‑point input)
- S‑AMB‑DG: Top floor T/RH (dew‑point input)

## Electrical and Meters

- R‑HP‑EN: Heat pump enable relay (from hydronic controller/safety chain)
- M‑HP‑EL: Electric sub‑meter for HP
- M‑HP‑HT: Heat meter HP→T1
- M‑FW‑HT: Heat meter T2→FW1 primary (optional)

## Control I/O Map (Example)

- Controller C‑HYD (hydronic):
  - Inputs: S‑T1‑TOP/MID/BOT, S‑T2‑TOP/MID/BOT, S‑UFH‑VL/RL, S‑AMB‑GF/DG, S‑UFH‑SURF (opt.)
  - Outputs: V‑MX‑UFH (0–10 V), P‑UFH (on/off or PWM), P‑FW‑PRI (via FW1), R‑HP‑EN, V‑SEAS‑T1/T2, alarm relay
  - Logic: dew‑point limit; DHW priority window; seasonal mode; anti‑short‑cycle; safe shutdown
- Controller C‑HP (in HP1):
  - Weather curve; supply/return temp limits; defrost; interface with R‑HP‑EN.
- FW1 internal controller:
  - Outlet temperature setpoint; modulates P‑FW‑PRI by flow/ΔT.

## Labeling and Documentation

- Each tag must appear on: hydraulic schematic, wiring diagram, device labels, and commissioning forms.
- Provide a printed tag legend and laminate a small copy near T1/T2 manifolds.
\end{markdown}

\chapter{Operating Guide}
\begin{markdown}
# Operating Guide — Daily Use and Seasonal Tips

This guide summarizes how to operate the system day‑to‑day and what to expect seasonally.

## Everyday

- DHW is on demand via FriWa; expect stable temperature at taps. Large simultaneous draws may cause a brief temperature dip; the DHW buffer will recover quickly.
- The system prefers midday DHW charging using PV. If weather is poor, DHW will still be maintained.
- Home Assistant shows buffer temps, humidity, and basic status; use notifications for filter service and unusual conditions.

## Winter (Heating)

- Weather‑compensated heating curve aims for low VL (~28–35°C). Radiant comfort is gradual but steady.
- Wood stove usage: enjoy as desired; it will lift buffer top temperature and the heat pump will trim back accordingly.
- If rooms feel cool, increase room setpoint slightly or raise the heating curve minimally; avoid large jumps.

## Summer (Cooling)

- Night charging chills the buffer to ~16–18°C. Daytime UFH supply is limited by dew‑point logic (typically 19–21°C); expect gentle background cooling.
- If indoor RH rises toward 60% and cooling pauses, ventilate during dry periods or use a temporary dehumidifier. This system intentionally avoids AC.

## Shoulder Seasons

- Minimal buffer temps; most comfort from passive gains and small heating boosts. DHW still prioritized around midday.

## Maintenance

- Filters: HRV and FriWa potable filter — inspect every 3–6 months (adjust to dust/water conditions).
- Visual checks: inspect for any condensation on cold manifolds/lines in early summer; increase dew‑point margin if needed.
- Annual service: verify safety valves, expansion vessel pressures, water quality (VDI 2035), heat meter readings.

## Troubleshooting Hints

- Hot water too cool: check DHW buffer temperature, FriWa outlet setting, and potable filter cleanliness.
- Cooling feels weak: check dew‑point limit vs. supply temp; if RH high, dry the air (ventilate when dry or dehumidify temporarily).
- Noise: enable night modes (HP and HRV), verify mounts and baffles.
\end{markdown}

\chapter{Costing — Order-of-Magnitude}
\begin{markdown}
# Costing — Order-of-Magnitude Ranges

Indicative ranges only; verify with quotes and current funding rules.

## Envelope

- Basement ceiling insulation: €20–90/m²
- Roof/attic insulation: €50–200/m²
- Triple-glazed windows incl. airtight install: €600–1,000 per window (or €280–900/m² window area)
- Facade insulation (WDVS): €90–210/m² (if feasible)
- Airtightness package + blower-door (pre/post): €1,500–4,200
- Stair partition (glazed): €1,000–3,000
- Loggia winter garden: €5,000–20,000 (design-dependent)
- Entry vestibule (glazed): similar to above, case-dependent

## Generation, Storage, Distribution

- Reversible AWHP (R290), installed: €18,000–35,000
- Heating/Cooling buffer 800–1000 L: €2,200–6,000
- DHW heating buffer 200–300 L: €800–1,800
- FriWa station: €1,700–3,500
- UFH retrofit (materials + install): €60–145/m²
- Hydronic periphery + full pipe insulation: €1,500–3,500
- Optional PHE (AWHP↔house): €800–2,700
- Wood stove with safety kit + flue works: €3,500–10,000 (+€800–3,000 chimney adjustments)

## Ventilation, PV, Electrical

- Decentralized HRV (6–8 units): €6,000–12,000
- PV 5.8–8 kWp: €8,000–12,000 (0% VAT) + grid fees
- Battery ~10 kWh with backup: €6,000–10,000 (0% VAT)
- Critical-loads subpanel + transfer: €1,500–3,000

## Soft Costs

- EEE (confirmations, site supervision): €2,000–5,000
- Technical planning (heat/cool load, schematics, LV, oversight): €3,000–8,000
- Commissioning and balancing protocols: €1,000–2,500

## Totals (Very Rough)

- Without facade: ~€67,000–90,000
- With facade: ~€82,000–115,000

Funding may reduce net costs (e.g., EH loan repayment bonus; Einzelmaßnahmen 15–20% with iSFP; PV/battery 0% VAT). Avoid double-funding and confirm all current terms prior to award.
\end{markdown}

\end{document}
