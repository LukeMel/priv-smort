\markdownRendererDocumentBegin
\markdownRendererSectionBegin
\markdownRendererHeadingOne{Operating Guide — Daily Use and Seasonal Tips}\markdownRendererInterblockSeparator
{}This guide summarizes how to operate the system day‑to‑day and what to expect seasonally.\markdownRendererInterblockSeparator
{}\markdownRendererSectionBegin
\markdownRendererHeadingTwo{Everyday}\markdownRendererInterblockSeparator
{}\markdownRendererUlBeginTight
\markdownRendererUlItem DHW is on demand via FriWa; expect stable temperature at taps. Large simultaneous draws may cause a brief temperature dip; the DHW buffer will recover quickly.\markdownRendererUlItemEnd 
\markdownRendererUlItem The system prefers midday DHW charging using PV. If weather is poor, DHW will still be maintained.\markdownRendererUlItemEnd 
\markdownRendererUlItem Home Assistant shows buffer temps, humidity, and basic status; use notifications for filter service and unusual conditions.\markdownRendererUlItemEnd 
\markdownRendererUlEndTight \markdownRendererInterblockSeparator
{}
\markdownRendererSectionEnd \markdownRendererSectionBegin
\markdownRendererHeadingTwo{Winter (Heating)}\markdownRendererInterblockSeparator
{}\markdownRendererUlBeginTight
\markdownRendererUlItem Weather‑compensated heating curve aims for low VL (~28–35°C). Radiant comfort is gradual but steady.\markdownRendererUlItemEnd 
\markdownRendererUlItem Wood stove usage: enjoy as desired; it will lift buffer top temperature and the heat pump will trim back accordingly.\markdownRendererUlItemEnd 
\markdownRendererUlItem If rooms feel cool, increase room setpoint slightly or raise the heating curve minimally; avoid large jumps.\markdownRendererUlItemEnd 
\markdownRendererUlEndTight \markdownRendererInterblockSeparator
{}
\markdownRendererSectionEnd \markdownRendererSectionBegin
\markdownRendererHeadingTwo{Summer (Cooling)}\markdownRendererInterblockSeparator
{}\markdownRendererUlBeginTight
\markdownRendererUlItem Night charging chills the buffer to ~16–18°C. Daytime UFH supply is limited by dew‑point logic (typically 19–21°C); expect gentle background cooling.\markdownRendererUlItemEnd 
\markdownRendererUlItem If indoor RH rises toward 60\markdownRendererUlItemEnd 
\markdownRendererUlEndTight \markdownRendererInterblockSeparator
{}
\markdownRendererSectionEnd \markdownRendererSectionBegin
\markdownRendererHeadingTwo{Shoulder Seasons}\markdownRendererInterblockSeparator
{}\markdownRendererUlBeginTight
\markdownRendererUlItem Minimal buffer temps; most comfort from passive gains and small heating boosts. DHW still prioritized around midday.\markdownRendererUlItemEnd 
\markdownRendererUlEndTight \markdownRendererInterblockSeparator
{}
\markdownRendererSectionEnd \markdownRendererSectionBegin
\markdownRendererHeadingTwo{Maintenance}\markdownRendererInterblockSeparator
{}\markdownRendererUlBeginTight
\markdownRendererUlItem Filters: HRV and FriWa potable filter — inspect every 3–6 months (adjust to dust/water conditions).\markdownRendererUlItemEnd 
\markdownRendererUlItem Visual checks: inspect for any condensation on cold manifolds/lines in early summer; increase dew‑point margin if needed.\markdownRendererUlItemEnd 
\markdownRendererUlItem Annual service: verify safety valves, expansion vessel pressures, water quality (VDI 2035), heat meter readings.\markdownRendererUlItemEnd 
\markdownRendererUlEndTight \markdownRendererInterblockSeparator
{}
\markdownRendererSectionEnd \markdownRendererSectionBegin
\markdownRendererHeadingTwo{Troubleshooting Hints}\markdownRendererInterblockSeparator
{}\markdownRendererUlBeginTight
\markdownRendererUlItem Hot water too cool: check DHW buffer temperature, FriWa outlet setting, and potable filter cleanliness.\markdownRendererUlItemEnd 
\markdownRendererUlItem Cooling feels weak: check dew‑point limit vs. supply temp; if RH high, dry the air (ventilate when dry or dehumidify temporarily).\markdownRendererUlItemEnd 
\markdownRendererUlItem Noise: enable night modes (HP and HRV), verify mounts and baffles.\markdownRendererUlItemEnd 
\markdownRendererUlEndTight 
\markdownRendererSectionEnd 
\markdownRendererSectionEnd \markdownRendererDocumentEnd